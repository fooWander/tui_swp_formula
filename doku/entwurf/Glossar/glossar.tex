\documentclass[fontsize = 12pt, paper = a4]{scrreprt} 

\setlength{\parindent}{0pt}
\usepackage[english,ngerman]{babel}
\usepackage[utf8]{inputenc}
\usepackage{enumerate}
\usepackage{amssymb,amsmath}

%------------ Überschriften verkleinern und hochsetzen ----------%

\makeatletter
\renewcommand*\@makechapterhead[1]{%
{\parindent \z@ \raggedright \normalfont
\LARGE\bfseries
\ifnum \c@secnumdepth >\m@ne
\thechapter\space
\fi
#1\par\nobreak
\vskip 20\p@
}} 

% ------------------------ Blattlayout- -------------------------%

\usepackage {geometry}   
\geometry   {left     = 2.5cm,
             right    = 2.5cm, 
             top      = 1.5cm,
             bottom   = 1.5cm,
             includehead, includefoot}
             
% ------------------------ Seitenstil ---------------------------%           

% Umdefinieren von Befehlen zur Vermeidung von Bugs:

\renewcommand*{\chapterpagestyle}{scrheadings} 
\renewcommand*{\chapterheadstartvskip}{\vspace*{-\topskip}}

% Gestaltung der Kopf- und Fußzeile:

\pagenumbering{arabic}
            
\usepackage[automark]{scrpage2}
\automark[chapter]{section}
\pagestyle{scrheadings} 
\ohead[\pagemark]{\pagemark}
\setlength{\footskip}{5mm} 

\clearscrheadfoot
\lohead{Entwurfsdokument}
\rohead{\headmark}
\lofoot{Softwareprojekt TU Ilmenau SS 2013}
\rofoot{\pagemark}

% Kopf- und Fußzeilenlinie:

\setheadsepline{.6pt} % Linie für Kopfzeile
\setfootsepline{.6pt} % Linie für Fußzeile 

% Für Unterstreichungen:

\usepackage[normalem]{ulem}

% Buchstabenglättung am Rand:
  
\usepackage {microtype}  

%-------------------------------------------------------------------%

% Für die Einbindung von Bildern:

\usepackage[pdftex]{graphicx} % .pdf, .png oder .eps möglich! pdftex
\usepackage{rotating}         % Grafiken rotieren

% Nutzung in drei Umgebungen möglich:

% (1) \begin{turn}{Winkel} ...  \end{turn}
% (2) \begin{sideways} ... \end{sideways} 90° im math. pos. Sinn
% (3) \begin{rotate}{Winkel} ... \end{rotate} 
%     ---> 90° im math. pos. Sinn, allerdings keine Platzreservierung 

\usepackage{wrapfig}
\usepackage{picinpar} % Textumflossene Grafiken
\usepackage{picins}


%-------------------------------------------------------------------%
 

% Packete für schöne Tabellen:

\usepackage{booktabs}
\usepackage{array}    % optional
\usepackage{tabularx} % optional

\usepackage[font=footnotesize,labelfont=bf,singlelinecheck=false,
            format=plain,,justification=justified,indention=0cm]                     {caption} 

\usepackage{setspace}

%--------------------  Anfang des Dokuments  -----------------------%

\begin{document}

%*******************************************************************%

% Entwurf Titelseite:



%*******************************************************************%

% --------------------- Inhaltsverzeichnis -----------------------%

 % Seitenumbruch

%--------------------------  Einleitung  ---------------------------%

\chapter*{Glossar}



\textbf{Ajax}
\hangindent+65pt \hangafter=1
\ \ \ \ \ \ \ \ \ Ajax ist ein Apronym für die Wortfolge „Asynchronous JavaScript and XML“. Es bezeichnet ein Konzept der asynchronen Datenübertragung zwischen einem Browser und dem Server.\\

\textbf{API}
\hangindent+65pt \hangafter=1
\ \ \ \ \ \ \ \ \ \ \ Eine API (applictation programming interface) ist eine Programmierschnitt\-stelle, die Funktionalitäten einer Software anderen Programmen zur Verfügung stellt.\\

\textbf{Architekturmuster}
\hangindent+65pt \hangafter=1 \\
Ein Architekturmuster ist eine aus Erfahrung entstandene und optimierte Schablone für häufig wiederkehrende Probleme, anhand derer man Software entwerfen kann.\\

\textbf{Balancing}
\hangindent+65pt \hangafter=1
\ Der Begriff Balancer, zu deutsch (Zellen-Ladungszustands-) Ausgleicher oder Ausgleichsregler, bezeichnet ein elektrisches Gerät, das die gleichmäßige Ladung aller Zellen innerhalb eines Akkupacks gewährleistet.\\

\textbf{bidirektional}
\hangindent+65pt \hangafter=1 \\
Datenübertragung, die in beide Richtungen funktioniert. \\

\textbf{boolean}
\hangindent+65pt \hangafter=1 \\
Ein Datentyp für die Darstellung der Werte "`wahr"\ und "`falsch"\.\\

\textbf{Browser}
\hangindent+65pt \hangafter=1
\ \ \ Ein spezielles Computerprogramm, das für die Darstellung von Webseiten oder Daten verwendet wird.\\

\textbf{Busarray}
\hangindent+65pt \hangafter=1
\ \ \ Ein Busarray ist eine gebündelte und zusammengefasste Anzahl von Bussen.\\

\textbf{CAN-Bus}
\hangindent+65pt \hangafter=1
\ Der CAN-Bus (Controlled Area Network) ist ein asynchrones, serielles Bussystem und gehört zu der Klasse der Feldbusse.\\

\textbf{CSS}
\hangindent+65pt \hangafter=1
\ \ \ \ \ \ \ \ \   Cascading Style Sheets sind eine deklarative Sprache für Stilvorlagen (engl. stylesheets) von strukturierten Dokumenten. Sie werden vor allem zusammen mit HTML eingesetzt.\\


\textbf{CSV}
\hangindent+65pt \hangafter=1
\ \ \ \ \ \ \ \ \ \  Das Dateiformat CSV steht für den englischen Begriff Comma-separated values (gelegentlich auch Character-separated values) und beschreibt den Aufbau einer Textdatei zur Speicherung oder zum Austausch einfach strukturierter Daten.\\

\textbf{DC}
\hangindent+65pt \hangafter=1
\ \ \ \ \ \ \ \ \ \ Die Abkürzung DC (direct current) steht für Gleichstrom.\\

\textbf{Dongle}
\hangindent+65pt \hangafter=1
\ \ \ \ \ Ein Gerät (hier: USB Stick), das an einen Computer oder ähnliches angeschlossen wird, um eine Verbindung zu kontrollieren. Dies dient zum Datenschutz und zur Kontrolle von gültigen Lizenzen.\\

\textbf{Embedded-PC}
\hangindent+65pt \hangafter=1 \\
Ein modular aufgebauter Industrierechner, dessen besonderer Fokus auf Kompaktheit liegt.\\

\textbf{Ethernet}
\hangindent+65pt \hangafter=1 
\ \ Übertragungstechnologie für Daten durch kabelgebundene Netze.\\

\textbf{Gierrate}
\hangindent+65pt \hangafter=1 
\ \ \ Gierrate bezeichnet die Geschwindigkeit der Drehung eines Fahrzeugs um die Hochachse.\\

\textbf{GRPS}
\hangindent+65pt \hangafter=1 
\ \ \ \ \ \ General Packet Radio Service ist eine Übertragungstechnologie für die Mobilfunkübertragung von Daten (langsamer als UMTS.\\

\textbf{GPS}
\hangindent+65pt \hangafter=1 
\ \ \ \ \ \ \  \ Global Positiong System ist eine Technologie zur Positionsbestimmung und Zeitmessung.\\

\textbf{GUI}
\hangindent+65pt \hangafter=1 
\ \ \ \ \ \ \ \ \ \   Das Graphical User Interface (zu deutsch: Benutzeroberfläche) ist die graf\-ische Schnittstelle zwischen Software und dem Benutzer.\\

\textbf{HTML}
\hangindent+65pt \hangafter=1
\ \ \ \ Hypertext Markup Language ist eine textbasierte Auszeichnungssprache zur Strukturierung von Inhalten wie Texten, Bildern und Hyperlinks in Dokumenten und im Internet.\\

\textbf{Interface}
\hangindent+65pt \hangafter=1
\ \ Schnittstelle, über die Daten von verschiedenen Komponenten ausgetauscht werden können.\\

\textbf{JavaScript}
\hangindent+65pt \hangafter=1
JavaScript ist eine Skriptsprache, die hauptsächlich für dynamisches HTML in Webbrowsern eingesetzt wird. \\

\textbf{Latenz}
\hangindent+65pt \hangafter=1
\ \ \ \ Laufzeit von Signalen, die sich aus der Differenz von dem Absenden und Ankommen von Daten ergibt.\\

\textbf{Matlab/Simulink}
\hangindent+65pt \hangafter=1 \\
Software, die hier verwendet wird, um die von der MicroAutoBox II bereitgestellten Daten zu verpacken und an den Embedded-PC weiterzuleiten.\\

\textbf{Matlab/Simulink-Block}
\hangindent+65pt \hangafter=1 \\
Mit der blockorientierten und graphischen Oberfläche von Simulink werden Gleichungen in Form von (Übertragungs-)Blöcken eingegeben und dargestellt.\\

\textbf{MicroAutoBox II}
\hangindent+65pt \hangafter=1 \\
Ein Echtzeitsystem, welches für schnelles Funktions-Prototyping in Fullpass- und Bypass-Szenarien geeignet ist.\\


\textbf{MySQL}
\hangindent+65pt \hangafter=1 
\ \ \  MySQL ist eines der weltweit am weitesten verbreiteten relationalen Datenbankverwaltungssysteme. Es bildet die Grundlage für viele dynamische Webauftritte.\\

\textbf{Nutzersystem}
\hangindent+65pt \hangafter=1 \\
Verwaltung von Personen und deren Rechte im Rechtesystem.\\


%\textbf{Orgware}
%\hangindent+65pt \hangafter=1 
%\ \ Aus dem Englischen stammender Begriff, der die Rahmenbedingungen bei IT-Produkten und deren Projektabwicklung beschreibt.

\textbf{PHP}
\hangindent+65pt \hangafter=1
\ \ \ \ \ \ PHP ist eine Skriptsprache, die hauptsächlich zur Erstellung dynamischer Webseiten oder Webanwendungen verwendet wird.\\

\textbf{PID-Regler}
\hangindent+65pt \hangafter=1 \\
Ein PID-Regler besteht aus 3 Teilen, einem P-Anteil, einem I-Anteil und einem D-Anteil. PI steht für proportional integral wirkend (wie beim PI-Regler) und D steht für differentiell wirkend.\\

\textbf{Recht}
\hangindent+65pt \hangafter=1 
\ \ \ \ \ \ \ hier: Synonym für Erlaubnis.\\

\textbf{Rechtesystem}
\hangindent+65pt \hangafter=1 \\
Vergabe von Rechten für definierte Aktionen.\\

\textbf{Sampling}
\hangindent+65pt \hangafter=1 \\
Sampling ist Englisch für "`Abtastung"\. Hierbei wird ein kontinuierliches Signal in festgelegten Zeitabständen abgetastet und der Wert bestimmt.\\

\textbf{Service-Interface}
\hangindent+65pt \hangafter=1 \\
Ein Service Interface ist eine Mensch-Fahrzeug-Schnittstelle, die es dem Menschen ermöglichen soll auf alle wichtigen Daten zuzugreifen.\\

\textbf{single}
\hangindent+65pt \hangafter=1 \\
Ein von der IEEE festgelegter Datentyp für die Darstellung von Gleitkommazahlen. Benötigt 32 Bit für die Zahlendarstellung.\\

\textbf{Smartphone}
\hangindent+65pt \hangafter=1 \\
Ein Smartphone (zu dt. intelligentes Telefon) beschreibt die aktuelle Gene\-ration von Mobilfunktelefonen, deren Eingabe auf einem berührungsempfindlichen Bildschirm erfolgt.\\

\textbf{Socket}
\hangindent+65pt \hangafter=1
\ \ \ Ein Socket ist eine plattformunabhängige Schnittstelle, mit der man ein Netzwerkprotokoll implementieren kann. Dies ermöglicht es Rechnern, über Systemgrenzen hinweg miteinander zu kommunizieren.\\

\textbf{SSL/TLS}
\hangindent+65pt \hangafter=1 
\ \ Secure Sockets Layer/Transport Layer Security (zu Deutsch: Transportsicherheitsschicht) TLS ist die Weiterentwicklung von SSL, einem Protokoll zur verschlüsselten Datenübertragung im Internet.\\

\textbf{Tablet-PC}
\hangindent+65pt \hangafter=1 
Ein tragbarer, flacher Computer in besonders leichter Ausführung mit einem Touchscreen-Display, wobei im Unterschied zu Notebooks auf eine physische Tastatur verzichtet wird.\\

\textbf{Timeout}
\hangindent+65pt \hangafter=1 
\ \ \ Eine festgelegte Zeitgröße, die eine Aktion auslöst, wenn für einen mit dem Timeout festgelegten Zeitraum bestimmte Ereignisse ausgeblieben sind.\\

\textbf{Torque Vectoring}
\hangindent+65pt \hangafter=1 \\
Mit dem "`gezielt eingesetztem Drehmoment" (so die wörtliche Übersetzung von "`Torque Vectoring") lenkt das Fahrzeug noch spontaner und direkter in die Kurve ein, außerdem bleibt es deutlich länger spurstabil.\\

\textbf{Touchscreen}
\hangindent+65pt \hangafter=1 \\
{Ein Touchscreen (zu dt. berührungsempfindlcher Bildschirm) beschreibt einen Bildschirm, auf dem gleichzeitig Inhalte angezeigt werden können und Eingaben am Telefon durch Berührung erfolgen können.\\

\textbf{UDP}
\hangindent+65pt \hangafter=1 
\ \ \ \ \ \ \ \ UDP ist ein verbindungsloses Protokoll für Datenübertragungen zwischen zwei Anwendungen durch ein Netzwerk hindurch.\\

\newpage

\textbf{Umrichter}
\hangindent+65pt \hangafter=1
Hierbei handelt es sich um einen Stromrichter, der aus einer Wechselspannung eine in der Frequenz und Amplitude unterschiedliche Wechselspannung generiert.\\

\textbf{UMTS}
\hangindent+65pt \hangafter=1
\ \ \ \ \ \ Universal Mobile Telecommunications System ist eine Übertragungstechnologie für die Mobilfunkübertragung von Daten (schneller als GPRS).\\

%\textbf{Webspace}
%\hangindent+65pt \hangafter=1
%Als Webspace wird der bei einem Anbieter gemietete, aus dem Internet zugängliche Speicher bezeichnet, der für die Aufbewahrung von Daten und Webseiten verwendet wird.\\

\textbf{weiche Echtzeit}
\hangindent+65pt \hangafter=1 \\
Das System soll Daten in einem festgelegten Zeitrahmen zur Verfügung stellen, es muss aber nicht gewährleistet werden, dass es das immer tut.\\







%------------------------  Randbedingungen  ------------------------%



\end{document}