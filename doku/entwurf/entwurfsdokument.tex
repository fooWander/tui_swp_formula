%---------------------------%
% Entwurf Pflichtenheft     %
% Autor: Sebastian Zehnter  %
%---------------------------%

%-------------------------%
% Dokumenteigenschaften   %
%-------------------------%

% Verwendung von KOMA-Script
% Eine Anleitung hierzu befindet sich in der Dropxbox!

\documentclass[fontsize = 12pt, paper = a4]{scrreprt} 

\setlength{\parindent}{0pt}
\usepackage[english,ngerman]{babel}
\usepackage[utf8]{inputenc}
\usepackage{enumerate}
\usepackage{amssymb,amsmath}

%------------ Überschriften verkleinern und hochsetzen ----------%

\makeatletter
\renewcommand*\@makechapterhead[1]{%
{\parindent \z@ \raggedright \normalfont
\LARGE\bfseries
\ifnum \c@secnumdepth >\m@ne
\thechapter\space
\fi
#1\par\nobreak
\vskip 20\p@
}} 

% ------------------------ Blattlayout- -------------------------%

\usepackage {geometry}   
\geometry   {left     = 2.5cm,
             right    = 2.5cm, 
             top      = 1.5cm,
             bottom   = 1.5cm,
             includehead, includefoot}
             
% ------------------------ Seitenstil ---------------------------%           

% Umdefinieren von Befehlen zur Vermeidung von Bugs:

\renewcommand*{\chapterpagestyle}{scrheadings} 
\renewcommand*{\chapterheadstartvskip}{\vspace*{-\topskip}}

% Gestaltung der Kopf- und Fußzeile:

\pagenumbering{arabic}
            
\usepackage[automark]{scrpage2}
\automark[chapter]{section}
\pagestyle{scrheadings} 
\ohead[\pagemark]{\pagemark}
\setlength{\footskip}{5mm} 

\clearscrheadfoot
\lohead{Pflichtenheft}
\rohead{\headmark}
\lofoot{Softwareprojekt TU Ilmenau SS 2013}
\rofoot{\pagemark}

% Kopf- und Fußzeilenlinie:

\setheadsepline{.6pt} % Linie für Kopfzeile
\setfootsepline{.6pt} % Linie für Fußzeile 

% Für Unterstreichungen:

\usepackage[normalem]{ulem}

% Buchstabenglättung am Rand:
  
\usepackage {microtype}  

% Für die Einbindung von Bildern:

\usepackage[pdftex]{graphicx} % Nur JPEGs möglich! 
%\usepackage{float}  

% Packete für schöne Tabellen:

\usepackage{booktabs}
\usepackage{array}    % optional
\usepackage{tabularx} % optional

\usepackage[font=footnotesize,labelfont=bf,singlelinecheck=false,
            format=plain,,justification=justified,indention=0cm]                     {caption} 

\usepackage{setspace}

%-----------------------------------------------------%

% Befehle:

% \hyperref[label_name]{''link text''}
% \url{http://www.wikibooks.org}
% \href{http://www.wikibooks.org}{Wikibooks home}

%-----------------------------------------------------%

% Packet für Glossar:

\usepackage{hyperref}
\usepackage{datatool}
\usepackage[xindy,nonumberlist]{glossaries} 

\makeglossaries

\newglossaryentry{Smartphone}{name=Smartphone ,description= {Ein Smartphone (zu dt. intelligentes Telefon) beschreibt die aktuelle Generation von Mobilfunktelefonen, deren Eingabe auf einem berührungsempfindlichen Bildschirm erfolgt},plural=Smartphones}

\newglossaryentry{Touchscreen}{name= Touchscreen,description={Ein Touchscreen (zu dt. berührungsempfindlcher Bildschirm) beschreibt einen Bildschirm, auf dem gleichzeitig Inhalte angezeigt werden können und Eingaben am Telefon durch Berührung erfolgen können }}

\newglossaryentry{Browser}{name=Browser ,description={Ein spezielles Computerprogramm, das für die Darstellung von Webseiten oder Daten verwendet wird }}

\newglossaryentry{Embedded-PC}{name=Embedded-PC ,description={Ein modular aufgebauter Indurstrierechner, dessen besonderer Fokus auf Kompaktheit liegt }}

\newglossaryentry{CAN-Bus}{name=CAN-Bus ,description={Der CAN-Bus (Controlled Area Network) ist ein asynchrones, serielles Bussystem und gehört zu der Klasse der Feldbusse }}

\newglossaryentry{Tablet-PC}{name=Tablet-PC ,description={ Ein tragbarer, flacher Computer in besonders leichter Ausführung mit einem Touchscreen-Display, wobei im Unterschied zu Notebooks auf eine physische Tastatur verzichtet wird }}

\newglossaryentry{MicroAutoBox II}{name=Micro\-Auto\-Box II ,description={ Ein Echtzeitsystem, welches für schnelles Funktions-Prototyping in Fullpass- und Bypass-Szenarien geeignet ist}}

\newglossaryentry{GUI}{name=GUI ,description={Das Graphical User Interface (zu deutsch: Benutzeroberfläche) ist die grafische Schnittstelle zwischen Software und dem Benutzer }}

\newglossaryentry{Nutzersystem}{name=Nutzersystem ,description={Verwaltung von Personen und deren Rechte im Rechtesystem }}


\newglossaryentry{Recht}{name=Recht ,description={hier: Synonym für Erlaubnis }}

\newglossaryentry{Rechtesystem}{name=Rechtesystem ,description={Vergabe von Rechten für definierte Aktionen }}

\newglossaryentry{Webspace}{name=Webspace ,description={Als Webspace wird der bei einem Anbieter gemietete, aus dem Internet zugängliche Speicher bezeichnet, der für die Aufbewahrung von Daten und Webseiten verwendet wird }}

\newglossaryentry{HTML}{name=HTML ,description={Hypertext Markup Language ist eine textbasierte Auszeichnungssprache zur Strukturierung von Inhalten wie Texten, Bildern und Hyperlinks in Dokumenten und im Internet }}

\newglossaryentry{CSS}{name=CSS ,description={Cascading Style Sheets sind eine deklarative Sprache für Stilvorlagen (engl. stylesheets) von strukturierten Dokumenten. Sie werden vor allem zusammen mit HTML eingesetzt }}

\newglossaryentry{PHP}{name=PHP ,description={PHP ist eine Skriptsprache, die hauptsächlich zur Erstellung dynamischer Webseiten oder Webanwendungen verwendet wird }}

\newglossaryentry{JavaScript}{name=JavaScript ,description={JavaScript ist eine Skriptsprache, die hauptsächlich für dynamisches HTML in Webbrowsern eingesetzt wird }}

\newglossaryentry{Ajax}{name=Ajax ,description={ Ajax ist ein Apronym für die Wortfolge „Asynchronous JavaScript and XML“. Es bezeichnet ein Konzept der asynchronen Datenübertragung zwischen einem Browser und dem Server }}

\newglossaryentry{MySQL}{name=MySQL ,description={MySQL ist eines der weltweit am weitesten verbreiteten relationalen Datenbankverwaltungssysteme. Es bildet die Grundlage für viele dynamische Webauftritte }}

\newglossaryentry{Service-Interface}{name=Service-Interface,description={Ein Service Interface ist eine Mensch-Fahrzeug-Schnittstelle, die es dem Menschen ermöglichen soll auf alle wichtigen Daten zuzugreifen}}

\newglossaryentry{Matlab/Simulink}{name=Matlab/Simulink ,description={Software, die hier verwendet wird, um die von der MicroAutoBox II bereitgestellten Daten zu verpacken und an den Embedded-PC weiterzuleiten }}

\newglossaryentry{weiche Echtzeit}{name=weiche Echtzeit ,description={Das System soll Daten in einem festgelegten Zeitrahmen zur Verfügung stellen, es muss aber nicht gewährleistet werden, dass es das immer tut }}

\newglossaryentry{Interface}{name= Interface,description={Schnittstelle, über die Daten von verschiedenen Komponenten ausgetauscht werden können }}

\newglossaryentry{Ethernet}{name=Ethernet  ,description={Übertragungstechnologie für Daten durch kabelgebundene Netze }}

\newglossaryentry{UMTS}{name=UMTS ,description={Universal Mobile Telecommunications System ist eine Übertragungstechnologie für die Mobilfunkübertragung von Daten (schneller als GPRS) }}

\newglossaryentry{GPRS}{name=GPRS ,description={General Packet Radio Service ist eine Übertragungstechnologie für die Mobilfunkübertragung von Daten (langsamer als UMTS) }}

\newglossaryentry{Latenz}{name=Latenz ,description={Laufzeit von Signalen, die sich aus der Differenz von dem Absenden und Ankommen von Daten ergibt }}

\newglossaryentry{bidirektional}{name=bidirektional ,description={Datenübertragung, die in beide Richtungen stattfindet }}

\newglossaryentry{Dongle}{name=Dongle ,description={Ein Gerät (hier: USB Stick), das an einen Computer oder ähnliches angeschlossen wird, um eine Verbindung zu kontrollieren. Dies dient zum Datenschutz und zur Kontrolle von gültigen Lizenzen  }}

\newglossaryentry{GPS}{name=GPS ,description={Global Positiong System ist eine Technologie zur Positionsbestimmung und Zeitmessung }}

\newglossaryentry{Timeout}{name=Timeout ,description={Eine festgelegte Zeitgröße, die eine Aktion auslöst, wenn für einen mit dem Timeout festgelegten Zeitraum bestimmte Ereignisse ausgeblieben sind }}

\newglossaryentry{API}{name=API ,description={Eine API (applictation programming interface) ist eine Programmierschnittstelle, die Funktionalitäten einer Software anderen Programmen zur Verfügung stellt }}

\newglossaryentry{CSV}{name=CSV ,description={Das Dateiformat CSV steht für den englischen Begriff Comma-separated values (gelegentlich auch Character-separated values) und beschreibt den Aufbau einer Textdatei zur Speicherung oder zum Austausch einfach strukturierter Daten }}

\newglossaryentry{Matlab/Simulink-Block}{name=Matlab/Simulink-Block , description={Mit der blockorientierten und graphischen Oberfläche von Simulink werden Gleichungen in Form von (Übertragungs-)Blöcken eingegeben und dargestellt }}

\newglossaryentry{Orgware}{name=Orgware ,description={Aus dem Englischen stammender Begriff der die Rahmenbedingungen bei IT-Produkten und deren Projektabwicklung beschreibt }}

\newglossaryentry{Balancing}{name=Balancing ,description={Der Begriff Balancer, zu deutsch (Zellen-Ladungszustands-) Ausgleicher oder Ausgleichsregler, bezeichnet ein elektrisches Gerät, das die gleichmäßige Ladung aller Zellen innerhalb eines Akkupacks gewährleistet }}

\newglossaryentry{PID-Regler}{name=PID-Regler ,description={Ein PID-Regler besteht aus 3 Teilen, einem P-Anteil, einem I-Anteil und einem D-Anteil. PI steht für proportional integral wirkend (wie beim PI-Regler) und D steht für differentiell wirkend }}

\newglossaryentry{Torque Vectoring}{name=Torque Vectoring ,description={Mit dem "`gezielt eingesetztem Drehmoment" (so die wörtliche Übersetzung von "`Torque Vectoring") lenkt das Fahrzeug noch spontaner und direkter in die Kurve ein, außerdem bleibt es deutlich länger spurstabil }}

\newglossaryentry{DC}{name=DC ,description={Die Abkürzung DC (direct current) steht für Gleichstrom }}

\newglossaryentry{Gierrate}{name=Gierrate ,description={Gierrate bezeichnet die Geschwindigkeit der Drehung eines Fahrzeugs um die Hochachse }}

\newglossaryentry{Umrichter}{name=Umrichter ,description={Hierbei handelt es sich um einen Stromrichter, der aus einer Wechselspannung eine in der Frequenz und Amplitude unterschiedliche Wechselspannung generiert }}


% \documentclass[fontsize = 12pt, paper = a4]{scrreprt} 

\setlength{\parindent}{0pt}
\usepackage[english,ngerman]{babel}
\usepackage[utf8]{inputenc}
\usepackage{enumerate}
\usepackage{amssymb,amsmath}

%------------ Überschriften verkleinern und hochsetzen ----------%

\makeatletter
\renewcommand*\@makechapterhead[1]{%
{\parindent \z@ \raggedright \normalfont
\LARGE\bfseries
\ifnum \c@secnumdepth >\m@ne
\thechapter\space
\fi
#1\par\nobreak
\vskip 20\p@
}} 

% ------------------------ Blattlayout- -------------------------%

\usepackage {geometry}   
\geometry   {left     = 2.5cm,
             right    = 2.5cm, 
             top      = 1.5cm,
             bottom   = 1.5cm,
             includehead, includefoot}
             
% ------------------------ Seitenstil ---------------------------%           

% Umdefinieren von Befehlen zur Vermeidung von Bugs:

\renewcommand*{\chapterpagestyle}{scrheadings} 
\renewcommand*{\chapterheadstartvskip}{\vspace*{-\topskip}}

% Gestaltung der Kopf- und Fußzeile:

\pagenumbering{arabic}
            
\usepackage[automark]{scrpage2}
\automark[chapter]{section}
\pagestyle{scrheadings} 
\ohead[\pagemark]{\pagemark}
\setlength{\footskip}{5mm} 

\clearscrheadfoot
\lohead{Entwurfsdokument}
\rohead{\headmark}
\lofoot{Softwareprojekt TU Ilmenau SS 2013}
\rofoot{\pagemark}

% Kopf- und Fußzeilenlinie:

\setheadsepline{.6pt} % Linie für Kopfzeile
\setfootsepline{.6pt} % Linie für Fußzeile 

% Für Unterstreichungen:

\usepackage[normalem]{ulem}

% Buchstabenglättung am Rand:
  
\usepackage {microtype}  

%-------------------------------------------------------------------%

% Für die Einbindung von Bildern:

\usepackage[pdftex]{graphicx} % .pdf, .png oder .eps möglich! pdftex
\usepackage{rotating}         % Grafiken rotieren

% Nutzung in drei Umgebungen möglich:

% (1) \begin{turn}{Winkel} ...  \end{turn}
% (2) \begin{sideways} ... \end{sideways} 90° im math. pos. Sinn
% (3) \begin{rotate}{Winkel} ... \end{rotate} 
%     ---> 90° im math. pos. Sinn, allerdings keine Platzreservierung 

\usepackage{wrapfig}
\usepackage{picinpar} % Textumflossene Grafiken
\usepackage{picins}


%-------------------------------------------------------------------%
 

% Packete für schöne Tabellen:

\usepackage{booktabs}
\usepackage{array}    % optional
\usepackage{tabularx} % optional

\usepackage[font=footnotesize,labelfont=bf,singlelinecheck=false,
            format=plain,,justification=justified,indention=0cm]                     {caption} 

\usepackage{setspace}

%--------------------  Anfang des Dokuments  -----------------------%

\begin{document}

%*******************************************************************%

% Entwurf Titelseite:



%*******************************************************************%

% --------------------- Inhaltsverzeichnis -----------------------%

 % Seitenumbruch

%--------------------------  Einleitung  ---------------------------%

\chapter*{Glossar}



\textbf{Ajax}
\hangindent+65pt \hangafter=1
\ \ \ \ \ \ \ \ \ Ajax ist ein Apronym für die Wortfolge „Asynchronous JavaScript and XML“. Es bezeichnet ein Konzept der asynchronen Datenübertragung zwischen einem Browser und dem Server.\\

\textbf{API}
\hangindent+65pt \hangafter=1
\ \ \ \ \ \ \ \ \ \ \ Eine API (applictation programming interface) ist eine Programmierschnitt\-stelle, die Funktionalitäten einer Software anderen Programmen zur Verfügung stellt.\\

\textbf{Architekturmuster}
\hangindent+65pt \hangafter=1 \\
Ein Architekturmuster ist eine aus Erfahrung entstandene und optimierte Schablone für häufig wiederkehrende Probleme, anhand derer man Software entwerfen kann.\\

\textbf{Balancing}
\hangindent+65pt \hangafter=1
\ Der Begriff Balancer, zu deutsch (Zellen-Ladungszustands-) Ausgleicher oder Ausgleichsregler, bezeichnet ein elektrisches Gerät, das die gleichmäßige Ladung aller Zellen innerhalb eines Akkupacks gewährleistet.\\

\textbf{bidirektional}
\hangindent+65pt \hangafter=1 \\
Datenübertragung, die in beide Richtungen funktioniert. \\

\textbf{boolean}
\hangindent+65pt \hangafter=1 \\
Ein Datentyp für die Darstellung der Werte "`wahr"\ und "`falsch"\.\\

\textbf{Browser}
\hangindent+65pt \hangafter=1
\ \ \ Ein spezielles Computerprogramm, das für die Darstellung von Webseiten oder Daten verwendet wird.\\

\textbf{Busarray}
\hangindent+65pt \hangafter=1
\ \ \ Ein Busarray ist eine gebündelte und zusammengefasste Anzahl von Bussen.\\

\textbf{CAN-Bus}
\hangindent+65pt \hangafter=1
\ Der CAN-Bus (Controlled Area Network) ist ein asynchrones, serielles Bussystem und gehört zu der Klasse der Feldbusse.\\

\textbf{CSS}
\hangindent+65pt \hangafter=1
\ \ \ \ \ \ \ \ \   Cascading Style Sheets sind eine deklarative Sprache für Stilvorlagen (engl. stylesheets) von strukturierten Dokumenten. Sie werden vor allem zusammen mit HTML eingesetzt.\\


\textbf{CSV}
\hangindent+65pt \hangafter=1
\ \ \ \ \ \ \ \ \ \  Das Dateiformat CSV steht für den englischen Begriff Comma-separated values (gelegentlich auch Character-separated values) und beschreibt den Aufbau einer Textdatei zur Speicherung oder zum Austausch einfach strukturierter Daten.\\

\textbf{DC}
\hangindent+65pt \hangafter=1
\ \ \ \ \ \ \ \ \ \ Die Abkürzung DC (direct current) steht für Gleichstrom.\\

\textbf{Dongle}
\hangindent+65pt \hangafter=1
\ \ \ \ \ Ein Gerät (hier: USB Stick), das an einen Computer oder ähnliches angeschlossen wird, um eine Verbindung zu kontrollieren. Dies dient zum Datenschutz und zur Kontrolle von gültigen Lizenzen.\\

\textbf{Embedded-PC}
\hangindent+65pt \hangafter=1 \\
Ein modular aufgebauter Industrierechner, dessen besonderer Fokus auf Kompaktheit liegt.\\

\textbf{Ethernet}
\hangindent+65pt \hangafter=1 
\ \ Übertragungstechnologie für Daten durch kabelgebundene Netze.\\

\textbf{Gierrate}
\hangindent+65pt \hangafter=1 
\ \ \ Gierrate bezeichnet die Geschwindigkeit der Drehung eines Fahrzeugs um die Hochachse.\\

\textbf{GRPS}
\hangindent+65pt \hangafter=1 
\ \ \ \ \ \ General Packet Radio Service ist eine Übertragungstechnologie für die Mobilfunkübertragung von Daten (langsamer als UMTS.\\

\textbf{GPS}
\hangindent+65pt \hangafter=1 
\ \ \ \ \ \ \  \ Global Positiong System ist eine Technologie zur Positionsbestimmung und Zeitmessung.\\

\textbf{GUI}
\hangindent+65pt \hangafter=1 
\ \ \ \ \ \ \ \ \ \   Das Graphical User Interface (zu deutsch: Benutzeroberfläche) ist die graf\-ische Schnittstelle zwischen Software und dem Benutzer.\\

\textbf{HTML}
\hangindent+65pt \hangafter=1
\ \ \ \ Hypertext Markup Language ist eine textbasierte Auszeichnungssprache zur Strukturierung von Inhalten wie Texten, Bildern und Hyperlinks in Dokumenten und im Internet.\\

\textbf{Interface}
\hangindent+65pt \hangafter=1
\ \ Schnittstelle, über die Daten von verschiedenen Komponenten ausgetauscht werden können.\\

\textbf{JavaScript}
\hangindent+65pt \hangafter=1
JavaScript ist eine Skriptsprache, die hauptsächlich für dynamisches HTML in Webbrowsern eingesetzt wird. \\

\textbf{Latenz}
\hangindent+65pt \hangafter=1
\ \ \ \ Laufzeit von Signalen, die sich aus der Differenz von dem Absenden und Ankommen von Daten ergibt.\\

\textbf{Matlab/Simulink}
\hangindent+65pt \hangafter=1 \\
Software, die hier verwendet wird, um die von der MicroAutoBox II bereitgestellten Daten zu verpacken und an den Embedded-PC weiterzuleiten.\\

\textbf{Matlab/Simulink-Block}
\hangindent+65pt \hangafter=1 \\
Mit der blockorientierten und graphischen Oberfläche von Simulink werden Gleichungen in Form von (Übertragungs-)Blöcken eingegeben und dargestellt.\\

\textbf{MicroAutoBox II}
\hangindent+65pt \hangafter=1 \\
Ein Echtzeitsystem, welches für schnelles Funktions-Prototyping in Fullpass- und Bypass-Szenarien geeignet ist.\\


\textbf{MySQL}
\hangindent+65pt \hangafter=1 
\ \ \  MySQL ist eines der weltweit am weitesten verbreiteten relationalen Datenbankverwaltungssysteme. Es bildet die Grundlage für viele dynamische Webauftritte.\\

\textbf{Nutzersystem}
\hangindent+65pt \hangafter=1 \\
Verwaltung von Personen und deren Rechte im Rechtesystem.\\


%\textbf{Orgware}
%\hangindent+65pt \hangafter=1 
%\ \ Aus dem Englischen stammender Begriff, der die Rahmenbedingungen bei IT-Produkten und deren Projektabwicklung beschreibt.

\textbf{PHP}
\hangindent+65pt \hangafter=1
\ \ \ \ \ \ PHP ist eine Skriptsprache, die hauptsächlich zur Erstellung dynamischer Webseiten oder Webanwendungen verwendet wird.\\

\textbf{PID-Regler}
\hangindent+65pt \hangafter=1 \\
Ein PID-Regler besteht aus 3 Teilen, einem P-Anteil, einem I-Anteil und einem D-Anteil. PI steht für proportional integral wirkend (wie beim PI-Regler) und D steht für differentiell wirkend.\\

\textbf{Recht}
\hangindent+65pt \hangafter=1 
\ \ \ \ \ \ \ hier: Synonym für Erlaubnis.\\

\textbf{Rechtesystem}
\hangindent+65pt \hangafter=1 \\
Vergabe von Rechten für definierte Aktionen.\\

\textbf{Sampling}
\hangindent+65pt \hangafter=1 \\
Sampling ist Englisch für "`Abtastung"\. Hierbei wird ein kontinuierliches Signal in festgelegten Zeitabständen abgetastet und der Wert bestimmt.\\

\textbf{Service-Interface}
\hangindent+65pt \hangafter=1 \\
Ein Service Interface ist eine Mensch-Fahrzeug-Schnittstelle, die es dem Menschen ermöglichen soll auf alle wichtigen Daten zuzugreifen.\\

\textbf{single}
\hangindent+65pt \hangafter=1 \\
Ein von der IEEE festgelegter Datentyp für die Darstellung von Gleitkommazahlen. Benötigt 32 Bit für die Zahlendarstellung.\\

\textbf{Smartphone}
\hangindent+65pt \hangafter=1 \\
Ein Smartphone (zu dt. intelligentes Telefon) beschreibt die aktuelle Gene\-ration von Mobilfunktelefonen, deren Eingabe auf einem berührungsempfindlichen Bildschirm erfolgt.\\

\textbf{Socket}
\hangindent+65pt \hangafter=1
\ \ \ Ein Socket ist eine plattformunabhängige Schnittstelle, mit der man ein Netzwerkprotokoll implementieren kann. Dies ermöglicht es Rechnern, über Systemgrenzen hinweg miteinander zu kommunizieren.\\

\textbf{SSL/TLS}
\hangindent+65pt \hangafter=1 
\ \ Secure Sockets Layer/Transport Layer Security (zu Deutsch: Transportsicherheitsschicht) TLS ist die Weiterentwicklung von SSL, einem Protokoll zur verschlüsselten Datenübertragung im Internet.\\

\textbf{Tablet-PC}
\hangindent+65pt \hangafter=1 
Ein tragbarer, flacher Computer in besonders leichter Ausführung mit einem Touchscreen-Display, wobei im Unterschied zu Notebooks auf eine physische Tastatur verzichtet wird.\\

\textbf{Timeout}
\hangindent+65pt \hangafter=1 
\ \ \ Eine festgelegte Zeitgröße, die eine Aktion auslöst, wenn für einen mit dem Timeout festgelegten Zeitraum bestimmte Ereignisse ausgeblieben sind.\\

\textbf{Torque Vectoring}
\hangindent+65pt \hangafter=1 \\
Mit dem "`gezielt eingesetztem Drehmoment" (so die wörtliche Übersetzung von "`Torque Vectoring") lenkt das Fahrzeug noch spontaner und direkter in die Kurve ein, außerdem bleibt es deutlich länger spurstabil.\\

\textbf{Touchscreen}
\hangindent+65pt \hangafter=1 \\
{Ein Touchscreen (zu dt. berührungsempfindlcher Bildschirm) beschreibt einen Bildschirm, auf dem gleichzeitig Inhalte angezeigt werden können und Eingaben am Telefon durch Berührung erfolgen können.\\

\textbf{UDP}
\hangindent+65pt \hangafter=1 
\ \ \ \ \ \ \ \ UDP ist ein verbindungsloses Protokoll für Datenübertragungen zwischen zwei Anwendungen durch ein Netzwerk hindurch.\\

\newpage

\textbf{Umrichter}
\hangindent+65pt \hangafter=1
Hierbei handelt es sich um einen Stromrichter, der aus einer Wechselspannung eine in der Frequenz und Amplitude unterschiedliche Wechselspannung generiert.\\

\textbf{UMTS}
\hangindent+65pt \hangafter=1
\ \ \ \ \ \ Universal Mobile Telecommunications System ist eine Übertragungstechnologie für die Mobilfunkübertragung von Daten (schneller als GPRS).\\

%\textbf{Webspace}
%\hangindent+65pt \hangafter=1
%Als Webspace wird der bei einem Anbieter gemietete, aus dem Internet zugängliche Speicher bezeichnet, der für die Aufbewahrung von Daten und Webseiten verwendet wird.\\

\textbf{weiche Echtzeit}
\hangindent+65pt \hangafter=1 \\
Das System soll Daten in einem festgelegten Zeitrahmen zur Verfügung stellen, es muss aber nicht gewährleistet werden, dass es das immer tut.\\







%------------------------  Randbedingungen  ------------------------%



\end{document}



%================================================================%           

\begin{document}

%----------------------------------------------------------------%

% Entwurf Titelseite:

\titlehead{\begin{center}
\textbf{\Huge Pflichtenheft}
\end{center}}
		   
\title{\gls{Service-Interface} \\ für ein Formula-Student-Fahrzeug}

\subtitle{Technische Universität Ilmenau \\
		  Softwareprojekt SS 2013 \\ Gruppe 19}			
		
\author{Christian Boxdörfer \\ Thomas Golda \\ Daniel Häger \\ 
		David Kudlek \\  Tom Porzig \\ Tino Tausch \\ 
		Tobias Zehner \\ Sebastian Zehnter}
		
\date{24.04.2013}	 
	  
\publishers{betreut durch \\ \vspace{1cm} Dr. Heinz-Dietrich Wuttke, TU Ilmenau \\ Oliver Dittrich, fachlicher Betreuer Team StarCraft e.V.}

\maketitle		

%*****************************************************************%

% --------------------- Inhaltsverzeichnis -----------------------%



\begin{spacing}{0.95} 
\tableofcontents
\setcounter{tocdepth}{4} % Anzeige bis Gliederungsstufe 4
\addtocontents{toc}{\protect\enlargethispage{2\baselineskip}} 
\end{spacing}


\newpage % Seitenumbruch

% ---------------------- Zielbestimmungen ------------------------%

\chapter{Zielbestimmungen}

Die Zielstellung dieses Softwareprojektes ist die Erstellung eines \gls{Service-Interface}s für ein Formula-Student-Fahrzeug, welches die zu übermittelnden Daten und Parameter des Fahrzeuges auswertet und diese anschließend in strukturierter und anschaulicher Form auf einer Weboberfläche darstellt. Die hierfür benötigten Daten werden von einer \gls{MicroAutoBox II} der Firma dSPACE bereitgestellt, die mittels der Software \gls{Matlab/Simulink} zu programmieren ist. An einer eigens definierten Schnittstelle erfolgt eine Weiterleitung dieser Daten an   
einen \gls{Embedded-PC}, welcher diese nach vorhergehender Auswertung  über ein \gls{GPRS}/\gls{UMTS}-Modul an einen Webserver sendet. Von dort aus können die gesammelten Daten von einem Benutzer über eine Webseite in visuell aufbereiteter Form abgerufen werden. Zudem ist es auch nach vorheriger Autorisierung gestattet, das Fahrzeug zu administrieren und durch Modifikation bestimmter Parameter auf der Weboberfläche Einfluss auf das Fahrverhalten und Eigenschaften des Fahrzeuges auszuüben.

%----------------------------------------------------------------%

\section{Obligatorische Kriterien}

%----------------------------------------------------------------%

\subsection{\gls{MicroAutoBox II}}

\begin{itemize}

\item \textit{Programmierung:} Die \gls{MicroAutoBox II} der Firma dSPACE ist mittels der proprietären Software \gls{Matlab/Simulink} zu programmieren, wobei neben der Funktionalität der Fokus auf der einfachen Erweiterbarkeit der Softwarelösung liegt. 

\item \textit{Datenpaket:} Die ausgelesenen Fahrzeugdaten werden mit einem Simulink-Modell vor der Übertragung an den \gls{Embedded-PC} zu einem einzelnen Datenpaket gebündelt. 

\item \textit{Schnittstelle:} Es ist mit Hilfe des dSPACE Platform \gls{API} Package eine geeignete Schnittstelle zwischen der \gls{MicroAutoBox II} und dem \gls{Embedded-PC} zu definieren und zu programmieren. 

%\newpage
 
\end{itemize}

%----------------------------------------------------------------%

\subsection{\gls{Embedded-PC}}

\begin{itemize}

\item \textit{Datenverarbeitung:} Die von der \gls{MicroAutoBox II} übertragenen Datenpakte sind vom \gls{Embedded-PC} auszuwerten. Dabei wird eine Plausibilitätsprüfung durchgeführt, so dass alle Werte im korrekten Wertebereich liegen. Darüber hinaus kann kein Fehler erkannt werden.

\item \textit{Datenübertragung:} Folgende ausgewertete Daten sind mittels einer \gls{GPRS}/\gls{UMTS}-Verbindung über das UDP-Protokoll an den Webserver in \glslink{weiche Echtzeit}{weicher Echtzeit} ($\leq 1$ Sekunde) zu übertragen:  \newpage

\end{itemize}

%-----------------------------------------------------------------%

\textbf{A) Allgemeine Fahrzeugdaten:} 

\begin{table}[h]
\caption{Allgemeine Fahrzeugdaten}

\begin{tabular}{ l | c | c | c }

\toprule[1.5pt]

\textbf{Parameter} & \textbf{Bitlänge} 
& \textbf{Anzahl} &\textbf{Regelung} \\ 

\midrule
Status der Notausfunktionen     & $1$   & 10 & --- \\ 
Temperatur                      & $16$  & 3  & --- \\
Geschwindigkeit                 & $16$  & 1  & --- \\ 
Gaswert                         & $16$  & 2  & --- \\ 
Gesamtspannung Akku             & $16$  & 1  & --- \\ 
Aktuelle Fahrzeugzeit           & $640$ & 1  & --- \\

\midrule
Gesamtanzahl benötigte Bits & \multicolumn{2}{c}{$\sum 762$ bit} \\

\bottomrule[1.5pt]

\end{tabular}

\label{Allgemeine Fahrzeugdaten}
\end{table}

\vspace{\baselineskip}

%-----------------------------------------------------------------%

\textbf{B) Akkudaten} 

\begin{table}[h]
\caption{Akkudaten}

\begin{tabular}{ l | c | c | c }

\toprule[1.5pt]
\textbf{Parameter} & \textbf{Bitlänge} & \textbf{Anzahl} & \textbf{Regelung} \\ 

\midrule
Alle Zelldaten                       & 16 & 144 & --- \\
Alle Temperaturen                    & 16 & 48  & --- \\
Maximale Zellspannung                & 16 & 1   & --- \\
Minimale Zellspannung                & 16 & 1   & --- \\
Gesamtspannung Akku                  & 16 & 1   & --- \\
Vorgegebener Strom zum Ladegerät     & 16 & 1   & --- \\
Vorgegebene Spannung zum Ladegerät   & 16 & 1   & --- \\
Zustand \gls{Balancing}              & 1  & 144 & --- \\

\midrule
Gesamtanzahl benötigte Bits & \multicolumn{2}{c}{$\sum 3296$ bit} \\

\bottomrule[1.5pt]

\end{tabular}

\label{Akkudaten}
\end{table}

\newpage

\vspace{\baselineskip}

%-----------------------------------------------------------------%

\textbf{C) Dynamische Daten} 

\begin{table}[h]
\caption{Dynamische Daten}

\begin{tabular}{ l | c | c | c }

\toprule[1.5pt]
\textbf{Parameter} & \textbf{Bitlänge} & \textbf{Anzahl} & \textbf{Regelung} \\ 

\midrule
Geschwindigkeit                         & 16 & 3 & --- \\
Beschleunigung (X-, Y- und Z-Achse)     & 16 & 3 & --- \\
\gls{Gierrate} der 3 Achsen                   & 16 & 3 & --- \\
Drehzahlen Räder                        & 16 & 4 & --- \\
Wassertemperatur                        & 16 & 2 & --- \\
Bremsdruck                              & 16 & 1 & --- \\
Bremskraft                              & 16 & 1 & --- \\
Bremsposition                           & 16 & 1 & --- \\
Federweg                                & 16 & 4 & --- \\
Gaspedalstellung                        & 16 & 2 & --- \\
Lenkwinkel                              & 16 & 1 & --- \\

\midrule
Gesamtanzahl benötigte Bits & \multicolumn{2}{c}{$\sum 400$ bit} \\

\bottomrule[1.5pt]

\end{tabular}

\label{Dynamische Daten}
\end{table}

\vspace{\baselineskip}

%-----------------------------------------------------------------%

\textbf{D) Fahrdynamikregelung} 

\begin{table}[h]
\caption{Fahrdynamikregelung}

\begin{tabular}{ l | c | c | c }

\toprule[1.5pt]
\textbf{Parameter} & \textbf{Bitlänge} & \textbf{Anzahl} & \textbf{Regelung} \\ 

\midrule
Antriebsschlupfregelung    & 16 & 1 & An/Aus bzw. $0-100\%$ \\
\gls{PID-Regler} Antriebsschlupf & 16 & 3 & --- \\
\gls{Torque Vectoring}           & 16 & 1 & An/Aus bzw. $0-100\%$ \\
Lenkwinkel                 & 16 & 1 & ---\\

\midrule
Gesamtanzahl benötigte Bits & \multicolumn{2}{c}{$\sum 96$ bit} \\

\bottomrule[1.5pt]

\end{tabular}

\label{Fahrdynamikregelung}
\end{table}

\newpage

\vspace{\baselineskip}

%-----------------------------------------------------------------%

\textbf{E) Motor- und \gls{Umrichter}daten} 

\begin{table}[h]
\caption{Motor- und Umrichterdaten}

\begin{tabular}{ l | c | c | c }

\toprule[1.5pt]
\textbf{Parameter} & \textbf{Bitlänge} & \textbf{Anzahl} & \textbf{Regelung} \\ 

\midrule
\gls{DC} Strom              & 16 & 1 & --- \\
\gls{DC} Spannung           & 16 & 1 & --- \\
Motortemperatur             & 16 & 8 & --- \\
Stromgrenze                 & 16 & 1 & ja \\
Maximalleistung             & 16 & 1 & $0-100\%$ \\
Lüfterdrehzahl              & 16 & 1 & --- \\
Lüfter                      & 16 & 1 & $0-100\%$ \\
Pumpe                       & 16 & 1 & $0-100\%$ \\
Wassertemperatur            & 16 & 1 & --- \\

\midrule
Gesamtanzahl benötigte Bits & \multicolumn{2}{c}{$\sum 256$ bit} \\

\bottomrule[1.5pt]

\end{tabular}

\label{Motor- und Umrichterdaten}
\end{table}

\vspace{\baselineskip}
%\vspace{1cm}
 %-----------------------------------------------------------------%
 
 Somit ergibt sich aus den Tabellen 1.1 - 1.5 für den Gesamtwert der Übertragungsdaten der folgende Wert:
 
 \begin{table}[h]
 \caption{Gesamtbilanz Übertragungsdaten}
 
 \begin{tabular}{ l | c }
 
 \toprule[1.5pt]
 \textbf{Kategorie} & \textbf{Anzahl [bit]} \\ 
 
 \midrule
 Allgemeine Fahrzeugdaten    & 762   \\
 Akkudaten                   & 3296  \\
 Dynamische Daten            & 400   \\
 Fahrdynamikregelung         & 96    \\
 Motor- und Umrichterdaten   & 256   \\
 
 
 \midrule
 Gesamtanzahl benötigte Bits & $\sum 4810$ \\
 
 \bottomrule[1.5pt]
 
 \end{tabular}
 
 %\label{Bilanz}
 \end{table} 
 
 \vspace{\baselineskip}

 \newpage

%-----------------------------------------------------------------%

\subsection{Weboberfläche und -technik}

Mithilfe modernster Webtechniken wie \gls{HTML}5, \gls{CSS}3, \gls{PHP} und \gls{JavaScript} / \gls{Ajax} soll die Aufbereitung der Fahrzeugdaten für den Benutzer erreicht werden.

\subsubsection{Oberfläche}

\begin{itemize}

\item \textit{Benutzerfreundlichkeit:}
Die Seite wird so aufgebaut und gestaltet, dass sie sich gut bedienen lässt und Änderungen an den Werten des Fahrzeugs ohne (web-)technische Kenntnisse intuitiv umgesetzt werden können.


\item \textit{Übersichtliche Darstellung:}
Die Fahrzeugdaten werden je nach Art und Umfang so aufbereitet und dargestellt, dass der Zusammenhang klar ersichtlich ist und der Benutzer auf einen Blick eine systematische Übersicht der Fahrzeugwerte erhält. Die Unterteilung entspricht dabei Punkt 1.1.2 A bis E.

\end{itemize}

%\newpage

%-----------------------------------------------------------------%

\subsubsection{Technik}

\begin{itemize}

\item \textit{Datenbank:}
Mithilfe einer \gls{MySQL}-Datenbank sollen die vom Fahrzeug zur Verfügung gestellten Daten der Webseite und somit dem Nutzer bereitgestellt werden.


\item \textit{Benutzerverwaltung:} 
Ein Benutzersystem mit unterschiedlichen \gls{Recht}en soll klar abgrenzen, welche Nutzer Daten verändern oder sie lediglich einsehen dürfen.


\item \textit{Sicherheitskriterien:}
Zum Schutz der Daten der Benutzer wird auf der Webseite eine Anmeldung mit Passwort eingerichtet. Die Übetragung der Daten über eine GPRS/UMTS-Verbindung erfolgt im Klartext.
 
\end{itemize} 

%-----------------------------------------------------------------%

\section{Optionale Kriterien}

\begin{itemize}

\item \textit{Darstellung auf mobilen Endgeräten:}



\begin{itemize}

\item[1)]Die Webseite soll so dargestellt werden, dass sie sich auch von Mobilfunkgeräten aus bedienen und verwenden lässt. Insbesondere liegt das Augenmerk auf der Darstellung der Inhalte auf \glspl{Smartphone}.

\item[2)] Die Webseite soll dahingehend optimiert werden, dass auch die Bedienung über ein Gerät ermöglicht wird, das mit einem \gls{Touchscreen} ausgestattet ist.

\end{itemize}

\item \textit{\gls{GPS}-Datenauswertung:}

Anhand der gegebenen \gls{GPS}-Positionsdaten soll es ermöglicht werden, den Fahrtverlauf zu verfolgen und auf einem geeigneten Kartenmaterial visuell darzustellen. \\ Die Darstellung soll im Web-\gls{Interface} platziert werden.

\end{itemize}

\newpage

%-----------------------------------------------------------------%

\section{Eingrenzende Kriterien}

\begin{itemize}

\item \textit{Internetverbindung:}

\begin{itemize}

\item[1)] 
Falls keine Verbindung aufgebaut werden kann oder diese unterbrochen wird, soll der letzte Stand der Daten auf der Webseite angezeigt werden. Dies umfasst zu dem den Zeitpunkt der letzten Aktualisierung der Daten. Liegt der Zeitpunkt länger als 10 Sekunden zurück, wird außerdem noch eine Warnmeldung auf der Webseite ausgegeben.

\item[2)] Eine Priorisierung der einzelnen Datengruppen A) - E) des Fahrzeuges (s. 1.1.2), um bei geringer Übertragungsrate lediglich die wichtigsten Fahrzeugdaten bzw. Fahrzeugparameter zu übertragen, ist aufgrund des knappen zeitlichen Rahmens nicht vorgesehen.


\end{itemize}

\item \textit{\gls{Embedded-PC}:}

\begin{itemize}

\item[1)] 
Die Datenübermittlung funktioniert nur über eine Internetverbindung. Die Daten sind nur über die Webseite verfügbar.

\item[2)]
Falls Verbindungsprobleme mit der Internetverbindung auftreten, werden keine Datensätze auf dem \gls{Embedded-PC} zwischengespeichert. Daten, die während eines Verbindungsabbruchs von der \gls{MicroAutoBox II} übermittelt werden, werden verworfen.

%\newpage

\end{itemize}

\item \textit{Webseite:}

\begin{itemize}

\item
Die Webseite sowie alle gesammelten Daten sind nur Mitgliedern von StarCraft e.V. zugänglich, bzw. allen, die von Team StarCraft e.V. dazu ermächtigt wurden. 



\end{itemize}

\end{itemize}


% ----------------------- Produkteinsatz -------------------------%

\chapter{Produkteinsatz}

\section{Anwendungsbereiche}

Das Produkt soll bei der Verwaltung und Überwachung eines Formula-Student-Fahrzeugs verwendet werden. Es soll sich insbesondere durch einen benutzerfreundlichen Aufbau und eine einfache und intuitive Bedienung auszeichnen.

%-----------------------------------------------------------------%

\section{Zielgruppen}

Die Software wird speziell für den Team StarCraft e.V. entwickelt und setzt somit Grundkenntnisse über die verschiedenen Arten von Fahrzeugdaten, die vom Fahrzeug übermittelt werden voraus. Weiterhin sollen auch von Team StarCraft e.V. berechtigte Personen Zugang zu den dargestellten Werten erhalten.

%-----------------------------------------------------------------%

\section{Betriebsbedingungen}

Für den Austausch der Daten zwischen dem Fahrzeug und der Datenbank auf dem Webserver sollte eine ungestörte Verbindung zum \gls{GPRS}/\gls{UMTS}-Netz gewährleistet werden, damit die zu übermittelnden Daten auch stets rechtzeitig zum Ablauf der Frist übermittelt werden können. Dies impliziert insbesondere keine völlige Abschottung des Fahrzeugs vom Mobilfunknetz. Tunnel, Bäume oder größere Hallen sollten nach Möglichkeit gemieden werden.


% ----------------------- Produktumgebung ------------------------%

\chapter{Produktumgebung}
\section{Software}

Das Produkt ist unabhängig von dem verwendeten Betriebssystem des Gerätes, solange einer der folgenden \gls{Browser} vorhanden ist:   
     
\begin {itemize}
\item Internet Explorer 10 oder neuer
\item Opera 12 oder neuer
\item Mozilla Firefox 20 oder neuer
\item Google Chrome 26 oder neuer
\item Safari 5 oder neuer
\end{itemize}


\section{Hardware}

Für eine Nutzung des Produktes ist folgende Hardware obligatorisch:
            
\begin{itemize}
\item Fahrzeug mit einer \gls{MicroAutoBox II} der Firma dSPACE
\item \gls{Dongle} der Firma dSPACE
\item \gls{Embedded-PC} "`M2M-PC NANO" der Firma  ADYNA Deutschland GmbH
\item Internetfähiger Rechner oder ein vergleichbares Gerät (\gls{Tablet-PC}, \gls{Smartphone} etc.) 
\end{itemize}

% ---------------------- Produktfunktionen  ----------------------%

\chapter{Produktfunktionen}

% Erläuterungen zur Nummerierung: 
% /[(W|E|M)(Hauptpunkt)(Punkt)(Unterpunkt)]/

% \textbf{Gesamtprodukt:} \\

Als Gesamtprodukt wird ein \gls{Service-Interface}  erstellt, das aus den Teilen Webseite, \gls{Embedded-PC}  und \gls{Matlab/Simulink} besteht und es ermöglicht, dass Nutzer dieses \gls{Interface}s Daten und Parameter in \glslink{weiche Echtzeit}{weicher Echtzeit} einsehen und ändern können, wenn die unter 3) beschrieben Bedingungen eingehalten werden. 

%-----------------------------------------------------------------%

\section{\gls{MicroAutoBox II}} 

\textbf{/M100/} 
\hangindent+50pt \hangafter=1
Es wird ein \gls{Matlab/Simulink-Block}  implementiert, der die einzelnen von der \gls{MicroAutoBox II} bereitgestellten Daten bündelt und über ein \gls{Interface}   für den \gls{Embedded-PC} via \gls{Ethernet}  bereitstellt. Dieser Block wird auf der  \gls{MicroAutoBox II} liegen. 


%-----------------------------------------------------------------%

\section{\gls{Embedded-PC}} 

\textbf{/E100/}
\hangindent+50pt \hangafter=1
\ Der \gls{Embedded-PC} ermöglicht eine \gls{bidirektional}e Verbindung zwischen \gls{MicroAutoBox II} und dem Webserver. \\


\textbf{/E110/} 
\hangindent+50pt \hangafter=1
\ Es ist möglich Daten von der \gls{MicroAutoBox II} anzufragen. Die Abfrage erfolgt im regelmäßigen Abstand von einer Sekunde \\


\textbf{/E120/} 
\hangindent+50pt \hangafter=1
\ Die Daten werden vor dem Senden in die jeweilige Richtung auf dem  
\gls{Embedded-PC} entsprechend aufbereitet. \\


\textbf{/E130/}  
\ Dabei wird immer auch eine Plausibilitätsüberprüfung  stattfinden. \\

\textbf{/E200/}
\hangindent+50pt \hangafter=1
\ Die Verbindung zwischen \gls{Embedded-PC} und Webserver erfolgt über eine \\ \gls{GPRS}/\gls{UMTS}-Verbindung . \\

\textbf{/E210/} 
\ \ Die geringen Datenübertragungsraten und hohen \gls{Latenz}en  werden berücksichtigt. \\

\textbf{/E300/} 
\hangindent+50pt \hangafter=1
\ Die Verbindung zwischen \gls{Embedded-PC} und \gls{MicroAutoBox II} erfolgt über eine \gls{Ethernet}-Verbindung. \newpage

%-----------------------------------------------------------------%

\section{Webseite} 

Es wird eine Webseite implementiert, die als \gls{GUI} für den Anwender dient. \\

\textbf{/W100/} 
\hangindent+50pt \hangafter=1
Die vom Fahrzeug empfangenen Daten werden auf der Webseite dargestellt und visuell aufbereitet. \\

\textbf{/W110/}
\hangindent+50pt \hangafter=1
Dazu werden diese in verschiedene Gruppen, wie in 1.1.2 bereits erwähnt, dargestellt. Ab einer Anzahl von fünf gleichartigen Daten werden diese tabellarisch wiedergegeben. \\

\textbf{/W120/} 
\hangindent+50pt \hangafter=1
Für eine detaillierte Darstellung der Daten wird es eine Möglichkeit der Nutzerinteraktion geben, die dazu führt, dass alle Daten dieser Gruppe aufgelistet werden. \\

\textbf{/W130/} 
\hangindent+50pt \hangafter=1
Es wird möglich sein, dass mehrere Benutzer (siehe /W500/) gleichzeitig verschiedene Daten betrachten können. Jedoch wird ausgeschlossen, dass mehr als ein Benutzer gleichzeitig die Möglichkeit besitzt die Fahrzeugparameter zu modifizieren. \\

\textbf{/W200/} 
\hangindent+50pt \hangafter=1
Es wird möglich sein, ausgewählte Parameter (siehe 1.1.2) zu ändern, die dann an die \gls{MicroAutoBox II} (siehe 3.2) gesendet werden. \\

\textbf{/W210/} 
\hangindent+50pt \hangafter=1
Durch ein \gls{Rechtesystem} (siehe /W520/) wird gewährleistet, dass nur von einer autorisierten Person ausgewählte Nutzer diese Parameter ändern dürfen und können. \\

\textbf{/W220/} 
\hangindent+50pt \hangafter=1
Zudem wird eine Plausibilitätsüberprüfung  durchgeführt, ob der gewählte Parameter den gewünschten Wert annehmen darf und ob er zum aktuellen Zeitpunkt überhaupt änderbar ist. Sollte ein ungültiger Wert entstehen, wird eine Fehlermeldung ausgegeben. Sollte der Wert zu einem ungültigen Zeitpunkt geändert werden, wird ebenfalls eine Fehlermeldung ausgegeben. \\


\textbf{/W300/}
\hangindent+50pt \hangafter=1
Die Fahrzeugdaten (siehe 1.1.2) werden über einen Zeitraum von mindestens zehn Stunden gespeichert. \\

\textbf{/W310/} 
\hangindent+50pt \hangafter=1
Es wird möglich sein, diese im \gls{CSV}-Format herunterzuladen. \\

\textbf{/W320/} 
\hangindent+50pt \hangafter=1
Solange eine Verbindung zwischen dem \gls{Embedded-PC} (siehe 2.3) und  dem Internet besteht, werden die von der \gls{MicroAutoBox II} zur Verfügung gestellten Daten gespeichert sowie die auf der Webseite geänderten Parameter an das Fahrzeug übertragen. \\

\textbf{/W400/} 
\hangindent+50pt \hangafter=1
Die Webseite wird auf einem von Team StarCraft e.V. gestellten Webserver laufen. \\

\newpage



\textbf{/W500/} 
\hangindent+50pt \hangafter=1
Es wird ein \gls{Nutzersystem}  implementiert. \\

\textbf{/W510/} 
\hangindent+50pt \hangafter=1
Dieses System wird die Registrierung (siehe /W530/) und Anmeldung \\ (siehe /W540/) von Anwendern ermöglichen. \\ 

\textbf{/W520/} 
\hangindent+50pt \hangafter=1
Ferner wird in das \gls{Nutzersystem} ein \gls{Rechtesystem} mit drei Arten von Nutzern \--- Vorstand, Techniker und Beobachter \--- implementiert. \\

\textbf{/W521/} 
\hangindent+50pt \hangafter=1
Der Vorstand wird die Möglichkeit haben, die \gls{Recht}e der Nutzer zu  bearbeiten sowie die Registrierung neuer Nutzer zu akzeptieren oder abzulehnen. Der Vorstand selbst hat die \gls{Recht}e eines Technikers (siehe /W522/). Ferner kann er bereits registrierte Nutzer löschen, was auf einer zusätzlichen Übersichtsseite möglich ist. \\

\textbf{/W522/} 
\hangindent+50pt \hangafter=1
Der Techniker wird die Möglichkeit haben, sowohl die Fahrzeugdaten auf der Webseite zu betrachten als auch Parameter ändern zu können. \\

\textbf{/W523/} 
\hangindent+50pt \hangafter=1
Der Beobachter wird nur die Möglichkeit haben, die Fahrzeugdaten auf der Webseite zu betrachten. \\

\textbf{/W530/} 
\hangindent+50pt \hangafter=1
Ein Nutzer wird sich mit einer E-Mail-Adresse und Angabe eines Passworts (mindestens sechs Stellen mit Ziffern) mit Wiederholung des selben registrieren können. Eine Registrierung wird zur Folge haben, dass der Vorstand durch die Anwendung darüber informiert wird. \\

\textbf{/W540/}
\hangindent+50pt \hangafter=1
Die Anmeldung auf der Webseite erfolgt mittels E-Mail-Adresse und Passwort. \\

\textbf{/W550/} 
\hangindent+50pt \hangafter=1
Es wird eine Option für das Rücksetzen des Passworts geben. \\


% ---------------------- Produktleistungen  ----------------------%

\chapter{Produktleistungen}

Das Produkt richtet sich sowohl an technische Laien, die den Verlauf beobachten sollen als auch an technisch versierte Nutzer, welche darüber hinaus die Fahrzeugwerte verändern dürfen. 

%-----------------------------------------------------------------%

\section{Benutzbarkeit}

\textbf{/L010/} 
\hangindent+45pt \hangafter=1
Das Produkt wird mit einer benutzerfreundlichen, grafischen Oberfläche versehen. \\

\textbf{/L020/} 
\hangindent+45pt \hangafter=1
Die Anwendung soll einfach zu bedienen sein. \\

\textbf{/L030/} 
\hangindent+45pt \hangafter=1
Das unberechtigte Verändern von kritischen Fahrzeugwerten ist nicht vorgesehen. \\

\textbf{/L040/} 
\hangindent+45pt \hangafter=1
Das Verändern von Fahrzeugwerten sollte nur durch Fachpersonal durchgeführt werden können. \\

\textbf{/L050/} 
\hangindent+45pt \hangafter=1
Fahrzeugwerte aus der Datenbank sollen komplett exportiert werden können, indem auf einen eigens dafür vorgesehenen Button geklickt wird. \\

\textbf{/L060/} 
\hangindent+45pt \hangafter=1
Der Zugang zur Webseite sollte von allen Endgeräten problemlos möglich sein (z.B. Computer, \gls{Tablet-PC}, \gls{Smartphone}). 

%-----------------------------------------------------------------%

\section{Robustheit}

\textbf{/L070/} 
\hangindent+45pt \hangafter=1
Wenn die Internetverbindung unterbrochen wird, soll sofort der Versuch gestartet werden, diese wieder aufzubauen. \\

\textbf{/L080/}
\hangindent+45pt \hangafter=1
Sollte der \gls{Embedded-PC} während des Betriebs neu starten, sollte die Funktionsfähigkeit hergestellt und die Ausführung begonnen bzw. fortgesetzt werden. \\

\textbf{/L090/} 
\hangindent+45pt \hangafter=1
Sollten viele Nutzer eine Verbindung zur Webseite aufbauen, soll diese weiterhin funktionsfähig und erreichbar bleiben. \\

\textbf{/L100/}
\hangindent+45pt \hangafter=1
Die Webseite sollte funktionsfähig und erreichbar bleiben, auch wenn viele Datensätze in die Datenbanken geschrieben werden. 

%-----------------------------------------------------------------%

\section{Echtzeiteigenschaften}

\textbf{/L110/} 
\hangindent+45pt \hangafter=1
Die Werte auf der Webseite sollen mindestens jede Sekunde aktualisiert werden, wenn dies nicht durch technische Umstände verhindert wird. \\

%-----------------------------------------------------------------%

\section{Effizienz}

\textbf{/L120/} 
\hangindent+45pt \hangafter=1
Die Übertragung über \gls{GPRS}/\gls{UMTS} sollte effizient gestaltet werden, sodass auch bei geringer Übertragungsrate die Funktionsfähigkeit gewährleistet ist. \\

\textbf{/L130/} 
\hangindent+45pt \hangafter=1
Die Anzeige der Werte und die Erstellung der Grafiken auf der Webseite sollte konform mit der Echtzeitanforderung sein. 

%-----------------------------------------------------------------%

\section{Erweiterbarkeit}

\textbf{/L140/}
\hangindent+45pt \hangafter=1
Die Erweiterbarkeit wird durch Open Source Programmierung gewährleistet. \\

\textbf{/L150/} 
\hangindent+45pt \hangafter=1
Eine gute Dokumentation ist ausschlaggebend für die Erweiterbarkeit.

%-----------------------------------------------------------------%

\section{Wartbarkeit}

\textbf{/L160/}
\hangindent+45pt \hangafter=1
Eine gute Dokumentation ist ebenfalls ausschlaggebend für die Wartbarkeit.

% --------------------- Qualitätsbestimmungen --------------------%

\chapter{Qualitätszielbestimmungen} 
\begin{table}[h]
\caption{Qualitätszielbestimmungen}

\begin{tabular}{ l | c | c | c | c | c}

\toprule[1.5pt]
\textbf{Qualitätsanforderung} & \textbf{sehr} & \textbf{wichtig} & \textbf{neutral} & \textbf{weniger} & \textbf{unwichtig} \\ 
& \textbf{wichtig} & & & \textbf{wichtig}  &\\

\midrule
Robustheit                     &    & \textbf{X}   &    &   & \\ %\hline
Zuverlässigkeit                &    &    &  \textbf{X}  &    &  \\ %\hline
Korrektheit                    & \textbf{X}   &    &     &    &  \\ %\hline
Benutzerfreundlichkeit         &    &  \textbf{X}  &    &    &  \\ %\hline
Effizienz                      &    & \textbf{X}   &    &    &  \\ %\hline
Portierbarkeit                 &    &    &    &  \textbf{X}  &  \\ %\hline
Kompatibilität                 &    &    &  \textbf{X}  &    &  \\

\bottomrule[1.5pt]

\end{tabular}


\end{table}

\vspace{\baselineskip}



% ------------------------ Testszenarien -------------------------%

\chapter{Testszenarien}

\section{\gls{MicroAutoBox II}}

\textbf{/TM100/} 
\hangindent+60pt \hangafter=1
Das auf der \gls{MicroAutoBox II} implementierte Simulink-Modell wird mittels proprietärer Software des Herstellers dSPACE auf eine vollständige Erfassung der Fahrzeugdaten am \gls{CAN-Bus} getestet. Hierbei soll auch verifiziert werden, dass diese korrekt gebündelt und an der \gls{Ethernet}-Schnittstelle bereitgestellt werden. 

\section{\gls{Embedded-PC}}

\textbf{/TE100/} 
\hangindent+60pt \hangafter=1
\ Bei einem hardware- oder softwaretechnischen Fehler soll der \gls{Embedded-PC} neu starten und danach sowohl die Verbindung zur \gls{MicroAutoBox II} und zur Webseite herstellen als auch den Zustand des Fahrzeuges überprüfen \\ (Stillstand oder in Bewegung, s. /TE110/). \\

\textbf{/TE110/} 
\hangindent+60pt \hangafter=1
\ Anhand des Umrichterschalters stellt der \gls{Embedded-PC} fest, ob sich das Fahrzeug im Stillstand oder in Bewegung befindet und verhindert in diesem Falle, dass während der Fahrt Änderungen an kritischen Fahrzeugparametern durchgeführt werden können. \\

\textbf{/TE120/} 
\hangindent+60pt \hangafter=1
\ Mittels einer Sammlung von Beispieldatensätzen wird in einer Simulation die korrekte Aufbereitung der Daten durch den \gls{Embedded-PC} überprüft. \\  

\textbf{/TE130/} 
\hangindent+60pt \hangafter=1
\ In einer Simulation soll der \gls{Embedded-PC} bei einer Plausibilitätsüberprüfung der ausgelesenen Daten  sowohl /TE110/ berücksichtigen als auch bei Statuswerten, die nicht in dem definierten Wertebereich der einzelnen Parameter liegen, eine Warnmeldung an die Webseite weiterleiten. Zudem soll der \gls{Embedded-PC} verhindern, dass eine Modifikation der veränderlichen Parameter über den festgelegten Wertebereich hinaus erfolgen kann. \\

\textbf{/TE200/} 
\hangindent+60pt \hangafter=1
\ Durch die Simulation verschiedener Szenarien (z.B. Funkloch etc.) wird die Stabilität und Signalqualität der \gls{GPRS}/\gls{UMTS}-Verbindung überprüft. \\

\textbf{/TE210/} 
\hangindent+60pt \hangafter=1
\ Der Webserver soll bei schlechter Signalqualität bzw. hohen \gls{Latenz}zeiten eine Statusmeldung auf der Webseite ausgeben. \\

\textbf{/TE300/} 
\hangindent+60pt \hangafter=1
\ Die \gls{Ethernet}-Schnittstelle zwischen \gls{MicroAutoBox II} und \gls{Embedded-PC} wird anhand einer Simulation auf deren korrekte Funktionsweise überprüft.  \\

\newpage

%-----------------------------------------------------------------% 

\section{Webseite}

\textbf{/TW100/TW110/} \\
\hangindent+65pt \hangafter=1
 \ Test auf eine korrekte Visualisierung  und Aufbereitung der Daten. \\ Hierbei wird zudem auf eine fehlerfreie Darstellung der Website bei einer Betrachtung dieser mit den im Folgenden aufgeführten \gls{Browser}n geachtet: \\
 
 \hangindent+65pt \hangafter=0
 1) Internet Explorer 10 \\
 2) Firefox 20 \\
 3) Opera 12 \\
 4) Chrome 26 \\
 5) Safari 5,6 \\
  

\textbf{/TW120/} 
\hangindent+65pt \hangafter=1
\ Bei der Nutzerinteraktion sollte die Möglichkeit bestehen die Daten in der Detailansicht fehlerfrei zu betrachten. \\

\textbf{/TW130/TW230/} \\
\hangindent+65pt \hangafter=1
\ Test des Verhaltens der Webseite bei simultanem Zugriff auf die Fahrzeugparameter. \\ %(Ausschlussprinzip o.Ä. bei Veränderung, bitte noch klären) \\ 

\textbf{/TW200/} 
\hangindent+65pt \hangafter=1
\ Die Kommunikation bei ausgeschaltetem \gls{Embedded-PC} schlägt fehl. Die Sendung soll bei einer Änderung der Parameter für eine bestimmte Zeit wiederholt werden. Nach dem Eintreten des \gls{Timeout}s erscheint eine Fehlermeldung. \\

\textbf{/TW210/TW521/TW522/TW523/} \\
\hangindent+65pt \hangafter=1
\ Ein Versuch eines Benutzers, Parameter des Fahrzeugs ohne die erforderlichen  \gls{Recht}e im \gls{Service-Interface} zu verändern, soll misslingen. \\

\textbf{/TW220/} 
\hangindent+65pt \hangafter=1
\ Die Modifizierung eines Parameters des Fahrzeuges außerhalb des festgelegten Rahmens sollte nicht möglich sein. Der Benutzer wird durch eine Fehlermeldung darauf hingewiesen. \\

\textbf{/TW300/} 
\hangindent+65pt \hangafter=1
\ Die Fahrzeugdaten werden über insgesamt 10 Stunden Laufzeit aufgenommen, wobei diese Daten verlustfrei gespeichert werden sollen. \\

\textbf{/TW310/} 
\hangindent+65pt \hangafter=1
\ Die gespeicherten Fahrzeugdaten werden als \gls{CSV}-Dateien exportiert und können fehlerfrei durch Microsoft Excel betrachtet und weiterverarbeitet werden. \\

\textbf{/TW320/TW400/} \\
\hangindent+65pt \hangafter=1
$\rightarrow$ siehe /TA400/ ! \\

\textbf{/TW320/} 
\hangindent+65pt \hangafter=1
\ Die Parameter des Fahrzeuges werden während der Fahrt geändert. Da dies   nicht möglich ist, wird der Benutzer mittels einer Fehlermeldung darüber informiert. \\


\textbf{/TW500/TW510/TW520/TW521/TW530/} \\
\hangindent+65pt \hangafter=1
\ Die Registrierung eines Nutzers mittels einer E-Mail Adresse und eines Passwortes(mindestens 6 Zeichen wie Buchstaben, Zahlen und Sonderzeichen) wird vorgenommen. Der Administrator wird durch die Anwendung über diese Registrierung informiert und kann den Benutzer freischalten oder abweisen, sowie ihn mit \gls{Recht}en versehen. \\
  
 %\newpage


\textbf{/TW521/} 
\hangindent+65pt \hangafter=1
\ Der Vorstand löscht einen Nutzer. Dieser Nutzer wird daraufhin aus der Datenbank entfernt. \\

\textbf{/TW522/} 
\hangindent+65pt \hangafter=1
\ Der Vorstand ändert die Parameter des Autos und greift auf die Fahrzeugdaten zu. Dies geschieht ohne Komplikationen. \\

\textbf{/TW523/} 
\hangindent+65pt \hangafter=1
\ Ein Benutzer mit den \gls{Recht}en eines Technikers  ändert die Parameter des Autos und greift auf die Fahrzeugdaten zu. Dies geschieht ohne Komplikationen. \\

\textbf{/TW524/} 
\hangindent+65pt \hangafter=1
\ Ein Benutzer mit den \gls{Recht}en eines Beobachters greift auf die Fahrzeugdaten zu. Dies geschieht ohne Komplikationen. Er hat nicht die Möglichkeit Parameter des Fahrzeuges zu ändern. \\

\textbf{/TW540/} 
\hangindent+65pt \hangafter=1
\hspace*{1mm}Ein registrierter Benutzer meldet sich mittels der bei der Registrierung angegebenen E-Mail-Adresse und seinem  Passwort an. Der Anmeldevorgang verläuft ohne Probleme. \\

\textbf{/TW541/} 
\hangindent+65pt \hangafter=1
\ Ein nichtregistrierter Nutzer versucht sich am System anzumelden. Der Anmeldevorgang schlägt fehl. \\
  
\vspace*{-0.5cm} 



%-----------------------------------------------------------------%

\section{Allgemeine Tests}

  \textbf{/TA100/} 
  \hangindent+65pt \hangafter=1
 \hspace*{0.1cm} Überprüfung der Systemreaktion bei einer simulierten Datenübertragung bzw. einer Modifikation der Fahrzeugparameter auf deren Korrektheit bei gleichzeitig auftretendem Stromausfall zur Vermeidung von Instabilitäten des Fahrzeuges. \\
  
  \textbf{/TA200/} 
  \hangindent+65pt \hangafter=1
 \hspace*{0.2cm} Bei einem Belastungstest der Serverdatenbanken sollte auch bei hohem Daten\-auf\-kom\-men jederzeit die Stabilität gewährleistet sein. \\
  
  \textbf{/TA300/} 
  \hangindent+65pt \hangafter=1
  \ \ Die Veränderung der Daten soll während der folgenden Szenarien  beobachtet werden: \\ \\ a) Im Funkloch (z.B. während einer Testfahrt) \\ b) Bei Stillstand des Fahrzeuges während der \gls{Embedded-PC} läuft \\
  
 
  \textbf{/TA400/} 
  \hangindent+65pt \hangafter=1
\ \ Das komplette System wird 2 Stunden lang im Normalbetrieb getestet. Dabei sollten keinerlei Komplikationen oder Fehler auftreten. 
 

% --------------------- Entwicklungsumgebung ---------------------%

\chapter{Entwicklungsumgebung}

\section{Software}

\begin{itemize}

\item \gls{MicroAutoBox II}

\begin{itemize}
\item \gls{Matlab/Simulink} R2011b \ \ $[$The MathWorks, Inc.$]$
\end{itemize}

\item \gls{Embedded-PC}

\begin{itemize}

\item Visual Studio 2012 Professional \ \ $[$Microsoft Corporation$]$

\item Eclipse Juno \ \ $[$Eclipse Foundation$]$

\item Notepad++ \ \ $[$Don Ho$]$

\item Sublime Text \ \ $[$Jon Skinner$]$

\item Git \ \ $[$Linus Torvalds$]$

\item GitHub \ \ $[$GitHub Inc.$]$

\end{itemize}

\item Webseite

\begin{itemize}

\item Xampp \ \ $[$Apache Friends$]$

\item FileZilla \ \ $[$Tim Kosse und Team$]$

\item Firefox \ \ $[$Mozilla Foundation$]$

\item Opera \ \ $[$Opera Software ASA$]$

\item Chrome \ \ $[$Google Inc.$]$

\item Safari \ \ $[$Apple Inc.$]$

\item Internet Explorer \ \ $[$Microsoft Corporation$]$

\item Notepad++ \ \ $[$D