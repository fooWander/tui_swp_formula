\documentclass[fontsize = 12pt, paper = a4]{scrreprt} 

\setlength{\parindent}{0pt}
\usepackage[english,ngerman]{babel}
\usepackage[utf8]{inputenc} 
\usepackage{enumerate}
\usepackage{amssymb,amsmath}

%------------ Überschriften verkleinern und hochsetzen ----------%

%\makeatlettern
%\renewcommand*\@makechapterhead[1]{%
%{\parindent \z@ \raggedright \normalfont
%\LARGE\bfseries
%\ifnum \c@secnumdepth >\m@ne
%\thechapter\space
%\fi
%#1\par\nobreak
%\vskip 20\p@
%}} 

% ------------------------ Blattlayout- -------------------------%

\usepackage {geometry}   
\geometry   {left     = 2.5cm,
             right    = 2.5cm, 
             top      = 1.5cm,
             bottom   = 1.5cm,
             includehead, includefoot}
             
% ------------------------ Seitenstil ---------------------------%           

% Umdefinieren von Befehlen zur Vermeidung von Bugs:

\renewcommand*{\chapterpagestyle}{scrheadings} 
\renewcommand*{\chapterheadstartvskip}{\vspace*{-\topskip}}

% Gestaltung der Kopf- und Fußzeile:

\pagenumbering{arabic}
            
\usepackage[automark]{scrpage2}
\automark[chapter]{section}
\pagestyle{scrheadings} 
\ohead[\pagemark]{\pagemark}
\setlength{\footskip}{5mm} 

\clearscrheadfoot
\lohead{Benutzerhandbuch}
\rohead{\headmark}
\lofoot{Softwareprojekt TU Ilmenau SS 2013}
\rofoot{\pagemark}

% Kopf- und Fußzeilenlinie:

\setheadsepline{.6pt} % Linie für Kopfzeile
\setfootsepline{.6pt} % Linie für Fußzeile 

% Für Unterstreichungen:

\usepackage[normalem]{ulem}

% Buchstabenglättung am Rand:
  
\usepackage {microtype} 

% Bildunterschriften zentrieren:

%\usepackage{caption}
%\captionsetup{margin=10pt,font=small,labelfont=bf, justification = centering}

%-------------------------------------------------------------------%

% Für die Einbindung von Bildern:

\usepackage[pdftex]{graphicx} % .pdf, .png oder .jpg möglich!
\usepackage{rotating}         % Grafiken rotieren

% Nutzung in drei Umgebungen möglich:

% (1) \begin{turn}{Winkel} ...  \end{turn}
% (2) \begin{sideways} ... \end{sideways} 90° im math. pos. Sinn
% (3) \begin{rotate}{Winkel} ... \end{rotate} 
%     ---> 90° im math. pos. Sinn, allerdings keine Platzreservierung 

\usepackage{wrapfig}
%\usepackage{picins}   % Textumflossene Grafiken
\usepackage{subfigure}
\usepackage{floatflt}
\usepackage[justification=centering]{caption}

%-------------------------------------------------------------------%
 
% Packete für Tabellen:

\usepackage{booktabs}
\usepackage{array}    % optional
\usepackage{tabularx} % optional

\usepackage[font=footnotesize,labelfont=bf,singlelinecheck=false,
            format=plain,,justification=justified,indention=0cm]                     {caption} 

\usepackage{setspace}

\usepackage{enumitem} 

%----------------  Anfang des Dokuments ------------------%

\begin{document}

%*******************************************************************%

% Entwurf Titelseite:

\titlehead{\begin{center}
\textbf{\Huge Entwurfsdokument}
\end{center}}
		   
\title{Service-Interface \\ für ein Formula-Student-Fahrzeug}

\subtitle{Technische Universität Ilmenau \\
		  Softwareprojekt SS 2013 \\ Gruppe 19}			
		
\author{Christian Boxdörfer \\ Thomas Golda \\ Daniel Häger \\ 
		David Kudlek \\  Tom Porzig \\ Tino Tausch \\ 
		Tobias Zehner \\ Sebastian Zehnter}
		
\date{Hier Datum einfügen}	 
	  
\publishers{betreut durch \\ \vspace{1cm} Dr. Heinz-Dietrich Wuttke, TU Ilmenau \\ Oliver Dittrich, fachlicher Betreuer Team StarCraft e.V.}

\maketitle		

%*******************************************************************%

% --------------------- Inhaltsverzeichnis -----------------------%

\begin{spacing}{0.86} 
\tableofcontents
%\setcounter{secnumdepth}{4} % Tiefere Gliederungsebene  
\setcounter{tocdepth}{4} % Anzeige bis Gliederungsstufe 4
%\addtocontents{toc}{\protect\enlargethispage{2\baselineskip}} 
\end{spacing}


\newpage % Seitenumbruch

%--------------------------  Einleitung  ---------------------------%

\chapter{Einleitung}

%----- Installation und Konfiguration des Service Interfaces -------%

\chapter{Installation und Konfiguration des Service Interfaces}


%-------------------------------------------------------------------%
%-------------------------------------------------------------------%

\section{MicroAutoBox II}

Für eine erfolgreiche Installation und Konfiguration der MicroAutoBox II müssen zu Beginn der Installation neben dieser Hardwarekomponente folgende Dateien in MATLAB und Modelle in Simulink vorliegen:

\begin{itemize}

\item \textit{udp\_final.mdl}: Diese Datei beinhaltet das von uns bereitgestellte Simulink-Modell für das Service Interface.

\item \textit{config\_datenpaket.m} Dieses *.m - File enthält die zur Konfiguration des Datenpaketes notwendigen Vektoren, welche je nach Art des Datenpaketes an dieses angepasst werden können und Informationen über dessen Attribute und Zusammensetzung beinhalten (Verweis ED).

\item \textit{signalgenerator\_microautobox.m} Dieses optionale *.m - File dient dazu, den Signalgenerator im Simulink-Modell zu Simulationszwecken mit generierten Testdaten auszustatten, um bei Veränderungen des Simulink-Modells oder bei einer Modifizierung der   auf dem Embedded-PC oder dem virtuellen Server implementierten *.cpp - Dateien eine Verifizierung des Service Interfaces anhand dieser bekannten Testdaten durchführen zu können (Verweis ED).

\end{itemize} 

Falls diese Dateien alle zur Verfügung stehen sollten, ist in einem ersten Schritt das Simulink-Modell \textit{udp\_final.mdl} durch das Programm MATLAB zu öffnen, wonach sich in Simulink auf der obersten Modellebene  folgende Subsysteme befinden (s. Abb. \ref{topmodell}):

\begin{figure}[h]
\centering
\includegraphics[scale = 0.65]{topmodell}
\caption[Gesamtaufbau Simulink-Modell]{Gesamtaufbau des Simulink-Modells auf höchster Modellebene}
\label{topmodell}
\end{figure} 

\newpage

%-------------------------------------------------------------------%

\subsection{Konfiguration der Ethernet-Schnittstelle}

Daraufhin ist bei der weiteren Vorgehensweise anschließend die Konfiguration der Ethernet-Schnittstelle vorzunehmen. Hierzu öffnet man durch einen Doppelklick den in Abb. \ref{topmodell} zu sehenden Block \textit{"`Ethernet UDP Setup"} ein Fenster, in welchem nun die Möglichkeit besteht, zwischen den beiden Reitern \textit{"`Unit"} und \textit{"`Options"} zu navigieren (Verweis dSPACE Doku) und dort bei den jeweiligen Einstellungen Modifikationen vorzunehmen. Im Folgenden werden obligatorische Änderungen durch ein (*) am jeweiligen Parameter gekennzeichnet. \\

\textbf{Reiter "`Unit"} 

\begin{itemize}

\item \textit{Interface Name}: Hier kann ein selbst gewählter Name für die Schnittstelle festgelegt werden.

\item \textit{$Board \ Type \ ^{(*)}$}: Bei Verwendung der MicroAutoBox II ist dort die Option \\ "`ETH Type 1"\ auszuwählen.

\item \textit{Module number}: Der dortige Wert ist auf "`1" vorkonfiguriert und kann auch so belassen werden.

\item \textit{$Local \ IP \ adress \ ^{(*)}$}: Hier ist die lokale IP-Adresse der MicroAutoBox II in Abhängigkeit des gewählten Subnetzes  anzugeben (z.B. 192.X oder 10.X).

\end{itemize}

\textbf{Reiter "`Options"} \\

In diesem Reiter können anhand nachfolgender Einstellungen bis zu vier verschiedene Sockets innerhalb des Modells definiert werden. Der Socket 1 ist hierbei für das Datenpaket mit den Fahrzeugdaten und der Socket 2 für das Datenpaket mit den Paketinformationen vorgesehen. Darüber hinaus stehen bei beabsichtigten Erweiterungen des Modells Socket 3 und 4 zur freien Verfügung.

\begin{itemize}

\item \textit{$Enable \ ^{(*)}$}: Ein gesetztes Häkchen entscheidet bei diesem Parameter darüber, ob der jeweilige Socket aktiviert oder deaktiviert wird. Es ist notwendig, die Sockets 1 und 2 zu aktivieren, um den Transport der Datenpakete an den Embedded-PC zu ermöglichen (s. o.). Darüber hinaus sollten die Sockets 3 und 4, falls diese nicht anderweitig verwendet werden, deaktiviert werden.  

\item \textit{$Local \ Port \ Number \ [0 \ ... \ 65535] \ ^{(*)}$}: In diesem Feld ist die Nummer des lokalen Ports der MicroAutoBox II einzutragen. 

\item \textit{$Remote \ Port \ Number \ [0 \ ... \ 65535] \ ^{(*)}$}:
Dort muss die Nummer des externen Ports -- also der gewünschte Port des Embedded-PCs -- eingetragen werden. \\ 

\textbf{Anmerkung:} Um Verwechslungen beim Eintragen der Portnummern o.ä. zu vermeiden, ist es empfehlenswert, für beide Ports die selbe Nummer zu vergeben. 

\end{itemize} 

\newpage

Nachdem alle obligatorischen Änderungen vorgenommen wurden, muss in einem nächsten Schritt innerhalb der Subsysteme \textit{UDP\_DATEN} und \textit{UDP\_PAKETINFORMATIONEN} die Blöcke "`ETHERNET\_UDP\_TX\_BL1"\ und "`ETHERNET\_UDP\_TX\_BL2"\ angepasst werden. Um zu diesen Blöcken zu gelangen, verfolgt man in bekannter Weise durch Doppelklicks auf der obersten Modellebene die folgenden Pfade im Modell: 

\begin{itemize}[leftmargin=*]

\item "`ETHERNET\_UDP\_TX\_BL1"\ : \\ Signaltransmitter\_Embedded\_PC $\rightarrow$ UDP\_DATEN 

\item "`ETHERNET\_UDP\_TX\_BL2"\ : \\ Signaltransmitter\_Embedded\_PC $\rightarrow$ UDP\_PAKETINFORMATIONEN

\end{itemize}

Nach dem Öffnen der Einstellungen der beiden Blöcke muss bei dem Parameter "`Maximum Message Size"\ der gleiche Wert eingetragen werden, der auch am Port "`Message Size"\ am jeweiligen Block anliegt. Sollten diese Werte nicht bekannt sein, so können diese an den beiden Displays "`DISPLAY\_MSGSIZE\_DATEN"\ und "`DISPLAY\_MSGSIZE\_PAKETINFO"\ abgelesen werden. Falls dies nach dem Starten der Simulation aufgrund von Fehlermeldungen nicht der Fall sein sollte, müssen die beiden Subsysteme UDP\_DATEN und UDP\_PAKETINFORMATIONEN kurzzeitig vom restlichen Modell abgetrennt / entfernt werden und die ehemals hinführenden Leitungen durch Terminatoren abgeschlossen werden. Anschließend kann der Wert bei Simulationsbeginn abgelesen, die neu eingefügten Terminatoren nach Beenden der Simulation wieder entfernt und die beiden Subsysteme erneut an das restliche Modell an den vorherigen Stellen angeschlossen werden. \\ 

Falls jedoch die genaue Anzahl / die Signalbreite an Fahrzeugdaten bzw. an Paketinformationen bekannt sein sollte, kann die Größe der "`Maximum Message Size"\ auf elegantere Weise ermittelt werden. Da die Werte der Fahrzeugdaten den Datentyp \textit{int16} aufweisen und die Paketdaten den Datentyp \textit{uint8} besitzen, müssen die jeweiligen Signalbreiten nur mit 2 bzw. 1 (Byte) multipliziert werden, um den gesuchten Wert korrekt zu ermitteln \\ (s. die Subsysteme MSGSIZE\_DATEN und MSGSIZE\_PAKETINFO). \\  

\textbf{Anmerkung}: Für weiterführende Informationen und Hinweise empfiehlt es sich, die Dokumentation der Firma dSPACE bzgl. des RTI Ethernet (UDP) Blocksets aufmerksam zu studieren. 

\newpage




%-------------------------------------------------------------------%

\subsection{Konfiguration der Matlabfiles \textit{signalgenerator\_microautobox.m} und \textit{config\_datenpaket.m}}

%-------------------------------------------------------------------%

Abhängig von den weiteren Absichten des Benutzers werden im Folgenden nun für diese Ziele die jeweiligen Vorgehensweisen ausführlich erläutert. 

\subsection{Testen des Simulink-Modells durch den Signalgenerator}

%-------------------------------------------------------------------%


\subsection{Anschluss des Simulink-Modells des Formula-Student-Fahrzeuges an das Simulink-Modell des Service Interfaces}

%-------------------------------------------------------------------%

\subsection{Implementierung des Modells auf der MicroAutoBox II}

%-------------------------------------------------------------------%

\subsection{Appendix: Hinzufügen, Entfernen oder Modifizieren von Signalen}

%-------------------------------------------------------------------%




\section{Embedded-PC}

\section{vServer}

\section{Datenbanken}

\subsection{Fahrzeugdatenbank}

\subsection{Benutzerdatenbank}



\newpage

% Text von Thomas eingepflegt, gez. Sebastian

\section{Webseite}

Zur uneingeschränkten Nutzung der Software müssen einige Voraussetzungen erfüllt sein: 

\begin{itemize}

\item Webserver 

\begin{itemize}

\item PHP Version 5.3 oder höher
\item mindestens eine (idealerweise zwei) MySQL-Datenbank (MySQL Version 5.3 oder höher)
\item SMTP-Server mit Authentifizierung
\item X MB freien Speicherplatz für Webseite

\end{itemize}

\end{itemize}

Im Folgenden werden alle nötigen Installationsschritte für die gesamte Software sowie die entsprechenden Konfigurationseinstellungen erläutert, welche zu Beginn getroffen werden müssen. \\

\textit{Webseite - Konfiguration} \\

Alle ausgelieferten Verzeichnisse und Dateien müssen in das Stammverzeichnis (s. Beschreibung Ihres Hostingangebotes) hochgeladen werden. Bevor Sie dies jedoch tun, müssen sie die Datei \textit{includes/config.php} anpassen. \\ 

\begin{figure}[h]
\centering
\includegraphics[scale = 0.50]{website_config}
\caption[Auschnitt aus der config.php - Datei]{Auschnitt aus der config.php - Datei}
\label{websiteconfig}
\end{figure} 

\newpage

Für Sie sind lediglich sechs Zeilen wichtig: 

\begin{itemize}[leftmargin=*]

\item \textit{\$dbhost}: Diese Einstellung ist auf "`localhost"\ gestellt. In den meisten Fällen ist dies die Standardeinstellung. Sollten Sie von Ihrem Provider explizit andere Angaben erhalten haben, dann ändern Sie dieses Feld. Sollten Sie keine Informationen erhalten haben, wird mit hoher Wahrscheinlichkeit "`localhost"\ die richtige Wahl sein. Sollte es zu Problemen kommen, setzen Sie sich bitte mit Ihrem Provider in Verbindung.


\item \textit{\$dbuname}: Hier fügen Sie in einfachen Anführungszeichen den Ihnen vom Provider mitgeteilte Zugangsnamen für den Datenbankserver ein.


\item \textit{\$dbpass}: Hier fügen Sie entsprechend das an Sie vergebene Passwort für die Datenbank ein.

\item \textit{\$dbname\_fd}: Sollten Sie vom Provider bereits eine Datenbank erhalten haben, so fügen Sie hier den Namen der Datenbank ein. Wenn Sie vollen Zugriff auf den Datenbankserver haben und selber Datenbanken anlegen können, so steht es Ihnen frei, wie Sie die Datenbank der Fahrzeugdaten bezeichnen möchten.


\item \textit{\$dbname\_ud}: Sollten Sie vom Provider bereits eine Datenbank erhalten haben, so fügen Sie hier den Namen der Datenbank ein. Wenn Sie vollen Zugriff auf den Datenbankserver haben und selber Datenbanken anlegen können, so steht Ihnen die Wahl frei, wie Sie die Datenbank der Nutzerdaten bezeichnen möchten.

\textbf{Anmerkung:} Beide Datenbanknamen können identisch sein, z.B. wenn Sie von Ihrem Provider nur eine erhalten haben sollten. 

\item \textit{\$mail}: Hier tragen Sie bitte die E-Mail-Adresse des Vorstandes ein. Alle eingehenden Registrierungsanfragen werden an diese E-Mail-Adresse weitergeleitet. Diese kann auch nach der Installtion noch angepasst werden. Alle anderen Werte müssen ab dem Beenden der Installation unverändert bleiben um die volle Funktionsfähigkeit gewährleisten zu können.

\end{itemize}

\newpage


\textit{Webseite - Installation} \\

Nachdem die Konfiguration erfolgreich durchgeführt wurde, führen Sie auf dem Server die Datei \textit{install.php} im Hauptverzeichnis aus und füllen das entsprechende Formular aus und schicken dieses ab. Anschließend werden alle nötigen Datenbanken und Tabellen erzeugt, sowie die Nutzergruppen und der Vorstandsaccount eingerichtet.

\begin{figure}[h]
\centering
\includegraphics[scale = 0.50]{install}
\caption[Einrichtung des Vorstand - Accounts]{Einrichtung des Vorstand - Accounts}
\label{installvorstand}
\end{figure} 


Sollten Sie von Ihrem Provider bereits Datenbanken erhalten haben, tritt nach der Installation ein Fehler mit der Meldung auf, dass die zu erstellende Datenbank bereits existiert. Dies ist normal und kein Grund zur Sorge. \\ \\

Die Installation ist nun vollständig. Bitte löschen Sie die \textit{install.php} unverzüglich vom Server um Missbrauch zu vermeiden. Sie können Sich nun mit den angegeben Zugangsdaten einloggen.




%----------------- Bedienung des Service-Interfaces ----------------%

\chapter{Bedienung des Service-Interfaces}

\section{Startseite / Verwaltung}

Über die Startseite loggen Sie sich mit den Zugangsdaten mit denen Sie sich registriert haben an. Ob Ihr Account bereits freigeschaltet wurde, erfahren Sie vom Vorstand. Nach dem Einloggen werden hier allgemeine Informationen dargestellt, wie z.B. eine Liste der sich zur Zeit online befindlichen Nutzer und eine Exportfunktion zum Extrahieren der Fahrzeugdaten aus der Datenbank. Als Vorstand erhalten Sie zudem noch eine Liste sämtlicher registrierter Nutzer und eine Möglichkeit Nutzer freizuschalten bzw. zu löschen und Rechtegruppen zu vergeben oder zu ändern.

\section{Nutzerverwaltung (Vorstand)}

Wenn Sie einen Nutzer bearbeiten wollen, muss stets eine der beiden Radioboxen aktiviert sein und im Textfeld seine ID-Nummer eingetragen werden. Möchten Sie einen Nutzer löschen, so wählen sie "`Löschen"\ , möchten Sie ihn jedoch aktivieren oder bearbeiten, so wählen Sie "`Aktivieren"\ aus. Mittels des Dropdown-Menüs können Sie dem Benutzer eine Rechtegruppe zuweisen. \\ \\
\textbf{Achtung:} Sie können sich nicht selbst löschen, dies muss ein anderer Nutzer mit Vorstandsrechten für Sie erledigen!

\section{CSV-Export}

Durch Auswahl einzelner Chechboxen können Sie sich die Daten des Fahrzeugs als \\ CSV-Datei herunterladen. Aus technischen Gründen kann es beim Auswählen mehrerer Boxen dazu kommen, dass Datensätze fehlen. Dies können Sie vermeiden, indem sie die Tabellen einzeln exportieren.

\section{Passwort ändern}

Diese Seite ermöglicht es Ihnen Ihr Passwort - beispielsweise nach einem Reset - zu ändern. Sie erreichen diese über die unter dem Hauptmenü befindlichen Link "`Passwort ändern".

\section{Passwort vergessen}

Auf der Startseite befindet sich ein Link "`Passwort vergessen". Klicken Sie auf ihn und geben Sie ihre Emailadresse ein. Bei erfolgreicher Änderung des Passworts erhalten Sie das neue Passwort per Mail zugeschickt. Andernfalls erscheint eine Fehlermeldung.

\newpage

\section{Menüleiste}

Über die Menüleiste können Sie die einzelnen Unterseiten aufrufen. Jede Unterseite stellt eine andere Gruppe von Fahrzeuginformationen dar (s. Pflichtenheft). \\ \\

Es ist zu Empfehlen sich nach jedem Besuch der Seite wieder auszuloggen.

\end{document}