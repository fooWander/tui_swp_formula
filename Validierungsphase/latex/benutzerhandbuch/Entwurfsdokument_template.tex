\documentclass[fontsize = 12pt, paper = a4]{scrreprt} 

\setlength{\parindent}{0pt}
\usepackage[english,ngerman]{babel}
\usepackage[utf8]{inputenc} 
\usepackage{enumerate}
\usepackage{amssymb,amsmath}

%------------ Überschriften verkleinern und hochsetzen ----------%

%\makeatlettern
%\renewcommand*\@makechapterhead[1]{%
%{\parindent \z@ \raggedright \normalfont
%\LARGE\bfseries
%\ifnum \c@secnumdepth >\m@ne
%\thechapter\space
%\fi
%#1\par\nobreak
%\vskip 20\p@
%}} 

% ------------------------ Blattlayout- -------------------------%

\usepackage {geometry}   
\geometry   {left     = 2.5cm,
             right    = 2.5cm, 
             top      = 1.5cm,
             bottom   = 1.5cm,
             includehead, includefoot}
             
% ------------------------ Seitenstil ---------------------------%           

% Umdefinieren von Befehlen zur Vermeidung von Bugs:

\renewcommand*{\chapterpagestyle}{scrheadings} 
\renewcommand*{\chapterheadstartvskip}{\vspace*{-\topskip}}

% Gestaltung der Kopf- und Fußzeile:

\pagenumbering{arabic}
            
\usepackage[automark]{scrpage2}
\automark[chapter]{section}
\pagestyle{scrheadings} 
\ohead[\pagemark]{\pagemark}
\setlength{\footskip}{5mm} 

\clearscrheadfoot
\lohead{Benutzerhandbuch}
\rohead{\headmark}
\lofoot{Softwareprojekt TU Ilmenau SS 2013}
\rofoot{\pagemark}

% Kopf- und Fußzeilenlinie:

\setheadsepline{.6pt} % Linie für Kopfzeile
\setfootsepline{.6pt} % Linie für Fußzeile 

% Für Unterstreichungen:

\usepackage[normalem]{ulem}

% Buchstabenglättung am Rand:
  
\usepackage {microtype} 

% Bildunterschriften zentrieren:

%\usepackage{caption}
%\captionsetup{margin=10pt,font=small,labelfont=bf, justification = centering}

%-------------------------------------------------------------------%

% Für die Einbindung von Bildern:

\usepackage[pdftex]{graphicx} % .pdf, .png oder .jpg möglich!
\usepackage{rotating}         % Grafiken rotieren

% Nutzung in drei Umgebungen möglich:

% (1) \begin{turn}{Winkel} ...  \end{turn}
% (2) \begin{sideways} ... \end{sideways} 90° im math. pos. Sinn
% (3) \begin{rotate}{Winkel} ... \end{rotate} 
%     ---> 90° im math. pos. Sinn, allerdings keine Platzreservierung 

\usepackage{wrapfig}
%\usepackage{picins}   % Textumflossene Grafiken
\usepackage{subfigure}
\usepackage{floatflt}
\usepackage[justification=centering]{caption}

%-------------------------------------------------------------------%
 
% Packete für Tabellen:

\usepackage{booktabs}
\usepackage{array}    % optional
\usepackage{tabularx} % optional

\usepackage[font=footnotesize,labelfont=bf,singlelinecheck=false,
            format=plain,,justification=justified,indention=0cm]                     {caption} 

\usepackage{setspace}

\usepackage{enumitem} 

% Für Codesnippets:

\usepackage{listings}

\usepackage{color}

\definecolor{mygreen}{rgb}{0,0.6,0}
\definecolor{mygray}{rgb}{0.5,0.5,0.5}
\definecolor{mymauve}{rgb}{0.58,0,0.82}

\lstset{ %
  backgroundcolor=\color{white},   % choose the background color; you must add \usepackage{color} or \usepackage{xcolor}
  basicstyle=\footnotesize,        % the size of the fonts that are used for the code
  breakatwhitespace=false,         % sets if automatic breaks should only happen at whitespace
  breaklines=true,                 % sets automatic line breaking
  captionpos=b,                    % sets the caption-position to bottom
  commentstyle=\color{mygreen},    % comment style
%  deletekeywords={...},            % if you want to delete keywords from the given language
  escapeinside={\%*}{*)},          % if you want to add LaTeX within your code
  extendedchars=true,              % lets you use non-ASCII characters; for 8-bits encodings only, does not work with UTF-8
  frame=single,                    % adds a frame around the code
  keepspaces=true,                 % keeps spaces in text, useful for keeping indentation of code (possibly needs columns=flexible)
  keywordstyle=\color{blue},       % keyword style
  language=Octave,                 % the language of the code
%  morekeywords={*,...},            % if you want to add more keywords to the set
  numbers=left,                    % where to put the line-numbers; possible values are (none, left, right)
  numbersep=5pt,                   % how far the line-numbers are from the code
  numberstyle=\tiny\color{mygray}, % the style that is used for the line-numbers
  rulecolor=\color{black},         % if not set, the frame-color may be changed on line-breaks within not-black text (e.g. comments (green here))
  showspaces=false,                % show spaces everywhere adding particular underscores; it overrides 'showstringspaces'
  showstringspaces=false,          % underline spaces within strings only
  showtabs=false,                  % show tabs within strings adding particular underscores
  stepnumber=2,                    % the step between two line-numbers. If it's 1, each line will be numbered
  stringstyle=\color{mymauve},     % string literal style
  tabsize=2,                       % sets default tabsize to 2 spaces
  title=\lstname                   % show the filename of files included with \lstinputlisting; also try caption instead of title
}





%----------------  Anfang des Dokuments ------------------%

\begin{document}
%\lstset{language= C++}  

%*******************************************************************%

% Entwurf Titelseite:

\titlehead{\begin{center}
\textbf{\Huge Benutzerhandbuch}
\end{center}}
		   
\title{Service-Interface \\ für ein Formula-Student-Fahrzeug}

\subtitle{Technische Universität Ilmenau \\
		  Softwareprojekt SS 2013 \\ Gruppe 19}			
		
\author{Christian Boxdörfer \\ Thomas Golda \\ Daniel Häger \\ 
		David Kudlek \\  Tom Porzig \\ Tino Tausch \\ 
		Tobias Zehner \\ Sebastian Zehnter}
		
\date{3.7.2013}	 
	  
\publishers{betreut durch \\ \vspace{1cm} Dr. Heinz-Dietrich Wuttke, TU Ilmenau \\ Oliver Dittrich, fachlicher Betreuer Team StarCraft e.V.}

\maketitle		

%*******************************************************************%

% --------------------- Inhaltsverzeichnis -----------------------%

\begin{spacing}{0.86} 
\tableofcontents
%\setcounter{secnumdepth}{4} % Tiefere Gliederungsebene  
\setcounter{tocdepth}{4} % Anzeige bis Gliederungsstufe 4
%\addtocontents{toc}{\protect\enlargethispage{2\baselineskip}} 
\end{spacing}


\newpage % Seitenumbruch

%--------------------------  Einleitung  ---------------------------%

\chapter{Einleitung}

Dieses Dokument wird Sie durch die Installation der 
einzelnen Komponenten führen, dabei werden Ihnen ausführlich 
die einzelnen Schritte aufgeführt, wie Sie die Komponenten 
auf der MicroAutoboxII, dem Embedded-PC und dem vServer 
einrichten. \\
Dazu orientiert sich das Dokument am Fluss der 
Daten von MicroAutobox II zur Webseite und geht nacheinander 
auf die Zwischenstationen Embedded-PC, vServer und Cronjob, 
Datenbank und Webseite ein, wobei Ihnen Screenshots und 
einzugebende Befehle aufgezeigt werden.

%----- Installation und Konfiguration des Service Interfaces -------%

\chapter{Installation und Konfiguration des Service Interfaces}


%-------------------------------------------------------------------%
%-------------------------------------------------------------------%

\section{MicroAutoBox II}

Für eine erfolgreiche Installation und Konfiguration der MicroAutoBox II müssen zu Beginn der Installation neben dieser Hardwarekomponente folgende Dateien in MATLAB und Modelle in Simulink vorliegen:

\begin{itemize}

\item \textit{udp\_final.mdl}: Diese Datei beinhaltet das von uns bereitgestellte Simulink-Modell für das Service Interface.

\item \textit{config\_datenpaket.m} Dieses \textit{*.m} - File enthält die zur Konfiguration des Datenpaketes notwendigen Vektoren, welche je nach Art des Datenpaketes an dieses angepasst werden können und Informationen über dessen Attribute und Zusammensetzung beinhalten (siehe Entwurfsdokument).

\item \textit{signalgenerator\_microautobox.m} Dieses optionale \textit{*.m} - File dient dazu, den Signalgenerator im Simulink-Modell zu Simulationszwecken mit generierten Testdaten auszustatten, um bei Veränderungen des Simulink-Modells oder bei einer Modifizierung der   auf dem Embedded-PC oder dem virtuellen Server implementierten \textit{*.cpp} - Dateien eine Verifizierung des Service Interfaces anhand dieser bekannten Testdaten durchführen zu können (siehe Entwurfsdokument).

\end{itemize} 

Falls diese Dateien alle zur Verfügung stehen sollten, ist in einem ersten Schritt das Simulink-Modell \textit{udp\_final.mdl} durch das Programm MATLAB zu öffnen, wonach sich in Simulink auf der obersten Modellebene  folgende Subsysteme befinden (s. Abb. \ref{topmodell}):

\begin{figure}[h]
\centering
\includegraphics[scale = 0.65]{topmodell}
\caption[Gesamtaufbau Simulink-Modell]{Gesamtaufbau des Simulink-Modells auf höchster Modellebene}
\label{topmodell}
\end{figure} 

\newpage

%-------------------------------------------------------------------%

\subsection{Konfiguration der Ethernet-Schnittstelle}

Daraufhin ist bei der weiteren Vorgehensweise anschließend die Konfiguration der Ethernet-Schnittstelle vorzunehmen. Hierzu öffnet man durch einen Doppelklick den in Abb. \ref{topmodell} zu sehenden Block \textit{"`Ethernet UDP Setup"} ein Fenster, in welchem nun die Möglichkeit besteht, zwischen den beiden Reitern \textit{"`Unit"} und \textit{"`Options"} zu navigieren (siehe dSPACE Dokumentation) und dort bei den jeweiligen Einstellungen Modifikationen vorzunehmen. Im Folgenden werden obligatorische Änderungen durch ein (*) am jeweiligen Parameter gekennzeichnet. \\

\textbf{Reiter "`Unit"} 

\begin{itemize}

\item \textit{Interface Name}: Hier kann ein selbst gewählter Name für die Schnittstelle festgelegt werden.

\item \textit{$Board \ Type \ ^{(*)}$}: Bei Verwendung der MicroAutoBox II ist dort die Option \\ "`ETH Type 1"\ auszuwählen.

\item \textit{Module number}: Der dortige Wert ist auf "`1" vorkonfiguriert und kann auch so belassen werden.

\item \textit{$Local \ IP \ adress \ ^{(*)}$}: Hier ist die lokale IP-Adresse der MicroAutoBox II in Abhängigkeit des gewählten Subnetzes  anzugeben (z.B. 192.X oder 10.X).

\end{itemize}

\textbf{Reiter "`Options"} \\

In diesem Reiter können anhand nachfolgender Einstellungen bis zu vier verschiedene Sockets innerhalb des Modells definiert werden. Der Socket 1 ist hierbei für das Datenpaket mit den Fahrzeugdaten und der Socket 2 für das Datenpaket mit den Paketinformationen vorgesehen. Darüber hinaus stehen bei beabsichtigten Erweiterungen des Modells Socket 3 und 4 zur freien Verfügung.

\begin{itemize}

\item \textit{$Enable \ ^{(*)}$}: Ein gesetztes Häkchen entscheidet bei diesem Parameter darüber, ob der jeweilige Socket aktiviert oder deaktiviert wird. Es ist notwendig, die Sockets 1 und 2 zu aktivieren, um den Transport der Datenpakete an den Embedded-PC zu ermöglichen (s. o.). Darüber hinaus sollten die Sockets 3 und 4, falls diese nicht anderweitig verwendet werden, deaktiviert werden.  

\item \textit{$Local \ Port \ Number \ [0 \ ... \ 65535] \ ^{(*)}$}: In diesem Feld ist die Nummer des lokalen Ports der MicroAutoBox II einzutragen. 

\item \textit{$Remote \ Port \ Number \ [0 \ ... \ 65535] \ ^{(*)}$}:
Dort muss die Nummer des externen Ports -- also der gewünschte Port des Embedded-PCs -- eingetragen werden. \\ 

\textbf{Anmerkung:} Um Verwechslungen beim Eintragen der Portnummern o.ä. zu vermeiden, ist es empfehlenswert, für beide Ports die selbe Nummer zu vergeben. 

\end{itemize} 

\newpage

Nachdem alle obligatorischen Änderungen vorgenommen wurden, muss in einem nächsten Schritt innerhalb der Subsysteme \textit{UDP\_DATEN} und \textit{UDP\_PAKETINFORMATIONEN} die Blöcke "`ETHERNET\_UDP\_TX\_BL1"\ und "`ETHERNET\_UDP\_TX\_BL2"\ angepasst werden. Um zu diesen Blöcken zu gelangen, verfolgt man in bekannter Weise durch Doppelklicks auf der obersten Modellebene die folgenden Pfade im Modell: 

\begin{itemize}[leftmargin=*]

\item "`ETHERNET\_UDP\_TX\_BL1"\ : \\ Signaltransmitter\_Embedded\_PC $\rightarrow$ UDP\_DATEN 

\item "`ETHERNET\_UDP\_TX\_BL2"\ : \\ Signaltransmitter\_Embedded\_PC $\rightarrow$ UDP\_PAKETINFORMATIONEN

\end{itemize}

Nach dem Öffnen der Einstellungen der beiden Blöcke muss bei dem Parameter \textit{"`Maximum Message Size"\ }der gleiche Wert eingetragen werden, der auch am Port \textit{"`Message Size"\ }am jeweiligen Block anliegt. Ist die genaue Anzahl / die Signalbreite an Fahrzeugdaten bzw. an Paketinformationen bekannt, so kann die Größe der \textit{"`Maximum Message Size"\ }auf folgende Weise ermittelt werden: \\


Da die Werte der Fahrzeugdaten nur den Datentyp \textit{int16}  aufweisen und die Paketdaten nur die Datentypen \textit{int16}  und \textit{uint8} besitzen, muss bei den Fahrzeugdaten die  Signalbreite nur mit 2 Byte multipliziert werden, um den gesuchten Wert korrekt zu ermitteln. (s. das Subsysteme MSGSIZE\_DATEN). Im Falle der Paketdaten kann diese durch den Vektor im Feld \textit{Datatype} im \textit{Encode32} - Block berechnet werden (s. hierzu Seite 8 und die Kodierung der Datentypen im \textit{*.m} - File \textit{config\_datenpaket.m}). \\  

\textbf{Anmerkung}: Für weiterführende Informationen und Hinweise empfiehlt es sich, die Dokumentation der Firma dSPACE bzgl. des RTI Ethernet (UDP) Blocksets aufmerksam zu studieren. \\

\newpage




%-------------------------------------------------------------------%

\subsection{Konfiguration der Matlabfiles \textit{signalgenerator\_microautobox.m} und \textit{config\_datenpaket.m}}

Die Konfiguration der beiden Configdateien in MATLAB ist entsprechend des Entwurfsdokumentes vorzunehmen, in dem dies ausführlich beschrieben wurde (s. Entwurfsdokument Punkt 2.1.2). Nachdem dies geschehen ist, müssen die Parameter der beiden Dateien noch vor dem Implementieren des Simulink-Modells auf der MicroAutoBox II in den Workspace von MATLAB geladen werden. Hierzu wechselt man in das Verzeichnis, in welchem die sich die Configfiles befinden (s. Abb. \ref{configfilespfad}). In diesem Fall wäre es der Pfad \\ \textit{C:\textbackslash Users \textbackslash Max Mustermann \textbackslash Eigene Dokumente \textbackslash Beispiel}


\begin{figure}[h]
\centering
\includegraphics[scale = 0.65]{configfilespfad}
\caption[Verzeichniswechsel in MATLAB]{Verzeichniswechsel in MATLAB}
\label{configfilespfad}
\end{figure} 

Anschließend bewirkt die Eingabe von \textit{"`signalgenerator\_microautobox"} und \textit{"`config\_datenpaket"} im "`Command Window", dass die Parameter der beiden Dateien in den Workspace von MATLAB geladen werden. Falls keine Tippfehler o.ä. aufgetreten sind, sollte nun im Workspace folgendes zu sehen sein: 

\begin{figure}[h]
\centering
\includegraphics[scale = 0.65]{configfilesworkspace}
\caption[Parameter im Workspace]{Im Workspace enthaltene Parameter nach der Ausführung der beiden *.m-Files}
\label{configfilesworkspace}
\end{figure}  

\newpage

\textbf{Anmerkung:} Die Parameter "`Datentypen"\ , "`Kommasetzung"\ und "`Paketteilung" wurden nachträglich eingefügt, um den \textit{Encode32} - Block beim Enkodieren die korrekte Reihenfolge der Datentypen bei den zu übertragenden Paketinformationen mitzuteilen. Dies wird im Enkoder durch folgenden Vektor bei \textit{Datatype} vorgenommen: 

\begin{center}
$[ 4, \ 4, \ 4, \ Datentypen, \ Paketteilung, \ Kommasetzung ]$ 
\end{center}


 


%-------------------------------------------------------------------%

Abhängig von den weiteren Absichten des Benutzers werden im Folgenden nun für diese Ziele die jeweiligen Vorgehensweisen ausführlich erläutert. 

\subsection{Testen des Simulink-Modells durch den Signalgenerator}

Mittels des Simulink-Subsystems des Signalgenerators können dem Simulink-Modell beliebige, virtuell simulierte Werte zur Verfügung gestellt werden. Vor allem zur Validierung des Systems ist dies sehr sinnvoll, da es nicht nötig ist, die MicroAutoBox II in das Fahrzeug einzubauen bzw. Peripherie (Sensoren etc.) daran anzuschließen. Auch trägt dieser Simulink-Block stark zur Erweiterbarkeit und Wartbarkeit des Modells bei.  

\subsubsection{Modifizieren der Simulationswerte}

Um die vorhandenen Simulationswerte zu verändern gibt es zwei Möglichkeiten. 
Als erste Option steht eine direkte Änderung der Werte  der Parameter in den jeweiligen Source-Blöcken zur Verfügung. 
Die Anordnung der Blöcke ist im Entwurfsdokument genau erläutert. 
Die zweite Möglichkeit wäre die gewünschten Werte im \textit{*.m}-File \textit{signalgenerator\_microautobox.m} über die dazugehörigen Parameter zu verändern. 
Der Aufbau und Funktion dieses Files wurden hinreichend im Entwurfsdokument sowie im vorherigen Abschnitt beschrieben.


\subsubsection{Veränderungen am Modell} 

Bei Veränderungen des Simulink-Modells, beispielsweise um neue Sensorwerte aufzunehmen oder nicht mehr benötigte zu entfernen, bietet es sich an, den Signalgenerator zu modifizieren und für Testzwecke zu verwenden. Dazu müssen die jeweiligen Werte aus dem Modell einschließlich des Signalgenerators entfernt oder hinzugefügt werden. Wichtig dabei ist, dass dies im kompletten Simulink-Model vorgenommen wird, da es sonst zu Fehlern kommt (s. Punkt 2.1.6). 


\subsubsection{Anschließen des Signalgenerator-Blockes an das Modell}

Falls sich im restlichen Modell keine Änderungen ergeben haben, stellt das erneute Anschließen des Signalgenerators kein großes Problem dar. 
Dieser muss einfach über eine Leitung mit dem Signalkollektor verbunden werden.  
Liegen jedoch Veränderungen vor, müssen diese, wie gerade erwähnt, komplett im Modell beachtet und bearbeitet werden. 
Dabei ist es wichtig, dass die Reihenfolge der Signale im Busarray analog zum Pflichtenheft erhalten bleibt (siehe 2.1.4), denn die Reihenfolge, in welcher die Signale über einen Bus-Creator gebündelt bzw. über einen Bus-Selector aufgespalten werden, spielt eine große Rolle. 
Die Signalreihenfolge kann innerhalb dieser Blöcke eingesehen und bearbeitet werden. 


\subsubsection{Verwenden des Signalgenerator-Blockes zu Testzwecken}

Ist das Modell nun komplett eingerichtet und mit dem Signalgenerator-Subsystem verbunden, ist es wichtig, dass die dazugehörige Datei \textit{signalgenerator\_microautobox.m} ausgeführt wurde, damit die Parameterwerte der Signale MATLAB bekannt sind. Nach der Implementierung des Modells (siehe 2.1.5) liegen die gewünschten Werte am Ausgang der MicroAutoBox II an und können an den Embedded-PC zur anschließenden Weiterverarbeitung der Fahrzeugdaten übertragen werden.


%-------------------------------------------------------------------%


\subsection{Anschluss des Simulink-Modells des Formula-Student-Fahrzeuges an das Simulink-Modell des Service Interfaces}

Um das Simulink-Modell des Formaula-Student-Fahrzeuges mit dem Simulink-Modell des Service Interfaces zu verbinden, sind folgende Schritte notwendig: \\

Zu Beginn muss als Vorbereitung das Subsystem \textit{Signalgenerator} vom Subsystem \textit{Signalcollector\_Embedded\_PC} getrennt und aus dem Modell entfernt werden. Des Weiteren empfiehlt es sich, die einzelnen Signale des Modells von Team StarCraft e.V. zu einem einzelnen Busarray zusammenzufassen und erst dann mit dem Busarray des Subsystems \textit{Signalcollector\_Embedded\_PC} zu verbinden.
Hierbei ist die Reihenfolge der im Pflichtenheft festgelegten Fahrzeugdaten einzuhalten (siehe Abb. \ref{anschlussbus}) , um mögliche Fehler zu vermeiden. 

\begin{figure}[h]
\centering
\includegraphics[scale = 0.45]{anschlussbus}
\caption[Einstellungen des Bus Creators]{Einstellungen des Bus Creators, um eine Verbindung der beiden Simulink-Modelle vorzunehmen}
\label{anschlussbus}
\end{figure} 

Wurde das Verbinden der beiden Modelle über ein Busarray korrekt durchgeführt, kann nunmehr mit Punkt 2.1.5 fortgefahren werden.


%-------------------------------------------------------------------%

\subsection{Implementierung des Modells auf der MicroAutoBox II}

Um das Simulink-Modell auf die MicroAutoBox II zu überspielen, müssen neben dem Modell auch die folgenden Dateien der Firma dSPACE in einem gemeinsamen Ordner liegen: 

\begin{itemize}

\item ds32encode.m
\item ds867c\_eth\_bit\_encoder\_sfcn.c
\item ds867c\_eth\_bit\_encoder\_sfcn.mexw32
\item ds867c\_eth\_encode32\_sfcn.c
\item ds867c\_eth\_encode32\_sfcn.mexw32

\end{itemize}

\textbf{Anmerkung:} Falls in einer späteren Erweiterung des Modells der UDP-Receive-Block benutzt werden sollte (dies wurde als alternatives Modell entworfen, aber aus Gründen der Robustheit wurde entgegen dem Feinentwurf auf eine bidirektionale Kommunikation verzichtet), so müssen darüber hinaus noch folgende Dateien dem Ordner hinzugefügt werden:

\begin{itemize}

% decoder *m. - file?

\item ds867c\_eth\_bit\_decoder\_sfcn.c
\item ds867c\_eth\_bit\_decoder\_sfcn.mexw32

\end{itemize}

Ist diese Voraussetzung erfüllt, so kann innerhalb von Simulink durch den Aufruf \\ \textit{Tools $\rightarrow$ Real-Time Workshop $\rightarrow$ Build Model} oder alternativ durch die Tastenkombination  \\ \textit{Strg + B} das Modell kompiliert und auf die MicroAutoBox II überspielt werden, was durchaus ein bis zwei Minuten in Anspruch nehmen kann. \\

\textbf{Anmerkung:} Sollten während des Kompilierens unerwartete Fehler wie z.B. folgende Meldung (s. Abb. ) auftreten, so liegt dies höchstwahrscheinlich an einer falsch eingestellten Message Size in den beiden UDP-Send-Blöcken \textit{ETHERNET\_UDP\_TX\_BL1} und \textit{ETHERNET\_UDP\_TX\_BL2} der Subsysteme \textit{UDP\_DATEN} und \textit{UDP\_PAKETINFORMATIONEN} (s. hierzu Punkt 2.1.1).

\begin{figure}[h]
\centering
\includegraphics[scale = 0.48]{fehlermsgsize}
\caption[Fehlermeldung bei falsch konfigurierter Message Size]{Fehlermeldung bei falsch konfigurierter Message Size}
\label{fehlermsgsize}
\end{figure} 

\newpage

%-------------------------------------------------------------------%

\subsection{Appendix: Wichtige Hinweise zu dem Hinzufügen, Entfernen oder Modifizieren von Signalen}

Dieser Abschnitt soll dem späteren Nutzer wichtige Hinweise geben, an welchen Stellen Änderungen notwendig werden, sobald das Modell durch das Hinzufügen, Entfernen oder Modifizieren von Signalen verändert werden sollte. \\

Es empfiehlt sich, die folgende Checkliste abzuarbeiten, um mögliche Fehler beim Kompilieren einzugrenzen:

\begin{itemize}

\item Ist das Simulink-Modell korrekt an den Signalkollektor angeschlossen, d.h. die festgelegte Reihenfolge im Pflichtenheft eingehalten?

\item Wurden alle bereitgestellten Signale des Signalkollektors genutzt und falls nicht, wurden diese durch Terminatoren abgeschlossen oder durch vorher definierte Constant-Blöcke belegt?

\item Wurde für das Signal die richtige Verstärkung gewählt?

\item Liegen am Encoder-Block notwendigerweise die Daten im Datentyp \textit{double} vor?

\item Wurden die Signale im Encoder-Block korrekt in die entsprechenden Datentypen gewandelt (s. S.8 oben)?

\item Wurden in der \textit{*.m} - File \textit{config\_datenpaket.m} die Vektoren korrekt an die Änderungen angepasst?

\item Wurde bei den UDP-Send-Blöcken die korrekte Message Size eingetragen?

\item Wurden bei den Einstellungen der Ethernet-Schnittstelle die korrekten Werte für die Remote IP etc. eingetragen?

\end{itemize}

\newpage

%-------------------------------------------------------------------%

\section{Embedded-PC}

Folgen Sie den Anweisungen im Benutzerhandbuch  um auf den Embedded PC zuzugreifen. Im ausgelieferten Zustand ist der Embedded-PC bereits funktionstüchtig. Eine Änderung der Konfiguration ist unter Umständen nur notwendig bei Modifikationen an der MicroAutoBox II oder des virtuellen Servers. Eine jederzeit aktuelle Installations- bzw. Konfigurationsanleitung finden Sie unter:

\vspace*{4mm}

 \begin{lstlisting}[frame=single]
 https://github.com/fooWander/tui_swp_formula.git 
 \end{lstlisting}

\vspace*{-2mm} 

\textbf{Punkt 1)} \\

Standardmäßig wird die UMTS-Verbindung und das Programm automatisch beim Start des PCs gestartet.
Dies erfolgt über zwei Einträge in der Datei \textit{rc.local}.
Die Datei \textit{rc.local} befindet sich unter folgendem Verzeichnis:

\vspace*{4mm}
\begin{lstlisting}[frame=single]
/etc/rc.local
\end{lstlisting} 
\vspace*{-2mm}

%\textbf{Punkt 2)} \\

Das erwähnte Programm heißt \textit{embpc} und liegt ausgehend vom Rootnutzer im nachfolgenden Verzeichnis:

\vspace*{4mm}
\begin{lstlisting}[frame=single]
~/tui_swp_formula/bin
\end{lstlisting} 
\vspace*{-2mm}


\subsection{Netzwerkanpassungen des Programms}

\textbf{Punkt 2)} \\

Änderungen von IP-Adressen oder Portnummern können Sie wie folgt durchführen:

\vspace*{4mm}
\begin{lstlisting}[frame=single]
cd ~/tui_swp_formula/src

nano common.h
\end{lstlisting} 
\vspace*{-2mm}

Die nun dargestellte Datei editieren Sie entsprechend Ihren Änderungen. Genauere Anweisungen hierzu finden Sie als Kommentare vor.
Nachfolgend notwendige Tastatureingaben:
Verwenden Sie zum Speichern der Datei die Tastenkombination \textit{STRG + o} und bestätigen Sie danach mit \textit{ENTER}. Mit der Tastenkombination \textit{STRG + x} schließen Sie die Datei.
Um das Programm nach vorgenommenen Netzwerk-Änderungen neu zu starten ist folgende Vorgehensweise einzuhalten.

\newpage

%\vspace*{4mm}
\begin{lstlisting}[frame=single]
killall embpc

cd ~/tui_swp_formula/src
	
make embpc %*-B*)

~/tui_swp_formula/bin/embpc 1>/dev/null &
\end{lstlisting} 
\vspace*{-2mm}

\textbf{Punkt 3)} \\	

Die Konfigurationsdatei für UMTS-Einstellungen befindet sich unter dem Verzeichnispfad:

\vspace*{4mm}
\begin{lstlisting}[frame=single]
/etc/wvdial.conf
\end{lstlisting} 
\vspace*{-2mm}

Sie können die SIM-Karten Einstellungen für die UMTS-Verbindung durch folgenden Befehl aufrufen und ändern:

\vspace*{4mm}
\begin{lstlisting}[frame=single]
nano /etc/wvdial.conf
\end{lstlisting} 
\vspace*{-2mm}

Nachfolgend notwendige Tastatureingaben: \\

Verwenden Sie zum Speichern der Datei die Tastenkombination \textit{STRG + o} und bestätigen Sie danach mit \textit{ENTER}. Mit der Tastenkombination \textit{STRG + x} schließen Sie die Datei. \\

Die Standardeinstellung der Konfigurationsdatei \textit{wvdial.conf} lautet:

\vspace*{4mm}
\begin{lstlisting}[frame=single]
[Dialer netzclub]
Init1 = ATZ
Init2 = AT&F
Init3 = ATQ0 V1 E1 S0=0 &C1 &D2 +FCLASS=0
Init4 = AT+CGDCONT=1,"IP","pinternet.interkom.de"
Stupid Mode = 1
Modem Type = Analog Modem
ISDN = 0
Phone = *99***1#
Modem = /dev/ttyUSB3
Username = netzclub
Password = netzclub
Baud = 460800
\end{lstlisting} 
\vspace*{-2mm}

Um eventuelle Probleme mit dem SIM-Karten-Pin auszuschließen, deaktivieren Sie zuvor die Pinabfrage mit Hilfe eines Mobiltelefons. Andernfalls wird das Hinzufügen der nachfolgend aufgeführten Zeilen in der Konfigurationsdatei \textit{wvdial.conf} nach den Standardeinstellungen notwendig. Der Platzhalter \textit{abcd} steht hierbei für Ihre Pinziffern und muss durch diese ersetzt werden.

\newpage

%\vspace*{4mm}
\begin{lstlisting}[frame=single]
[Dialer PIN]
Init4 = AT+CPIN="abcd"
\end{lstlisting} 
\vspace*{-2mm}

Im Anschluss an die Konfiguration sollte der Embedded-PC neu gestartet werden. Der Befehl dazu befindet sich unter Punkt 5). \\

\textbf{Punkt 4)} \\

Bei Problemen kann die UMTS Verbindung nach einem Neustart der Netzwerkverbindungen manuell gestartet werden.
Dazu geben Sie zuerst folgenden Befehl für den Neustart der Netzwerkverbindungen ein:

\vspace*{4mm}
\begin{lstlisting}[frame=single]
/etc/rc.d/init.d/network restart 
\end{lstlisting} 
\vspace*{-2mm}

Nach diesem Befehl können Sie die UMTS-Verbindung über einen der folgenden Befehle manuell starten:

\vspace*{4mm}
\begin{lstlisting}[frame=single]
pppd call gsm   
\end{lstlisting} 
\vspace*{-2mm}

oder

\vspace*{4mm}
\begin{lstlisting}[frame=single]
wvdial netzclub 
\end{lstlisting} 
\vspace*{-2mm}

oder 

\vspace*{4mm}
\begin{lstlisting}[frame=single]
cd ~/tui_swp_formula

source startUMTS.sh	
\end{lstlisting} 
\vspace*{-2mm}

\textbf{Punkt 5)} \\
	
Der Embedded-PC kann durch den folgenden Befehl neu gestartet werden:

\vspace*{4mm}
\begin{lstlisting}[frame=single]
shutdown -r now
\end{lstlisting} 
\vspace*{-2mm}

Um den PC herunterzufahren wird der nachstehende Befehl genutzt:

\vspace*{4mm}
\begin{lstlisting}[frame=single]
shutdown -h now	
\end{lstlisting} 
\vspace*{-2mm}

\newpage

\subsection{Vorbereitung bei einer Neuinstallation eines Betriebssystems}

\textbf{Punkt 6)} \\

Bei einer Neuinstallation des Systems ist folgende Befehlsfolge einzuhalten um die erforderlichen Komponenten zu installieren (RPM basierte Distribution z.B. Fedora, Cent OS, etc.):

\vspace*{4mm}
\begin{lstlisting}[frame=single]
yum update

yum install wvdial usb-modeswitch ntp git ssh nano

yum groupinstall "Development Tools"
\end{lstlisting} 
\vspace*{-2mm}

Bei anderen Distributionen ist die Befehlsfolge an den entsprechenden Paketmanager und die Paketnamen anzupassen. \\

\textbf{Punkt 7)} \\

Eine Zeitsynchronisation ist wie folgt durchzuführen:
Den NTP-Dienst aktivieren Sie mit: 

\vspace*{4mm}
\begin{lstlisting}[frame=single]
chkconfig ntpd on
\end{lstlisting} 
\vspace*{-2mm}

Die Zeit können Sie nun mit dem Zeitserver \textit{0.europe.pool.ntp.org} wie folgt synchronisieren:

\vspace*{4mm}
\begin{lstlisting}[frame=single]
ntpdate 0.europe.pool.ntp.org
\end{lstlisting} 
\vspace*{-2mm}

Zuletzt wird der NTP-Dienst über folgende Eingabe gestartet: 

\vspace*{4mm}
\begin{lstlisting}[frame=single]
/etc/init.d/ntpd start
\end{lstlisting} 
\vspace*{-2mm}

Die Zeit kann über nachfolgenden Befehl kontrolliert werden:

\vspace*{4mm}
\begin{lstlisting}[frame=single]
date
\end{lstlisting} 
\vspace*{-2mm}

\textbf{Punkt 8)} \\

Um eine UMTS-Konfigurationsdatei zu erzeugen, benutzen Sie das nachfolgende Kommando:

\vspace*{3mm}
\begin{lstlisting}[frame=single]
wvdialconf /etc/wvdial.conf
\end{lstlisting} 
\vspace*{-2mm}

Die erstellte Konfigurationsdatei ist entsprechend Punkt 3) anzupassen.

\newpage

\textbf{Punkt 9)} \\

Die aktuelle Programmversion können Sie wie folgt über eine bestehende Internetverbindung herunterladen und installieren:

\vspace*{4mm}
\begin{lstlisting}[frame=single]
cd ~/
 
git clone git://github.com/fooWander/tui_swp_formula.git

cd tu-ilmenau/src/

make embpc %*-B*)
\end{lstlisting} 
\vspace*{-2mm}

\textbf{Punkt 10)} \\

Standardmäßig wird die UMTS-Verbindung automatisch beim Start des PCs über ein Skript hergestellt. Das erwähnte Skript heißt \textit{startUMTS.sh} welches in der Datei  \textit{rc.local } eingetragen ist. Die Datei \textit{rc.local} befindet sich unter folgendem Verzeichnis:

\vspace*{4mm}
\begin{lstlisting}[frame=single]
/etc/rc.local
\end{lstlisting} 
\vspace*{-2mm}

\textbf{Punkt 11)} \\

Um die UMTS-Verbindung automatisch bei Systemstart herzustellen, muss wie folgt vorgegangen werden:

\vspace*{4mm}
\begin{lstlisting}[frame=single]
echo ~/tui_swp_formula/startUMTS.sh >> /etc/rc.local
\end{lstlisting} 
\vspace*{-2mm}

\textbf{Punkt 12)} \\

Standardmäßig wird das Programm automatisch beim Start des PCs gestartet. Die Datei \textit{rc.local} befindet sich unter folgendem Verzeichnis:

\vspace*{4mm}
\begin{lstlisting}[frame=single]
/etc/rc.local
\end{lstlisting} 
\vspace*{-2mm}

\textbf{Punkt 13)} \\

Um das Programm automatisch bei Systemstart auszuführen, muss auch hier folgendes in die Konsole eingegeben werden:

\vspace*{4mm}
\begin{lstlisting}[frame=single]
echo ~/tui_swp_formula/bin/embpc >> /etc/rc.local
\end{lstlisting} 
\vspace*{-2mm}

\textbf{Punkt 14)} \\

Nachfolgend sind die Befehle für die notwendige Konfiguration der Firewall beschrieben: \\
Öffnen Sie die Firewall-Einstellung:

\vspace*{4mm}
\begin{lstlisting}[frame=single]
nano /etc/sysconfig/iptables
\end{lstlisting} 
\vspace*{-2mm}

Fügen Sie folgende Zeile an das Ende der Einstellung hinzu:

\vspace*{4mm}
\begin{lstlisting}[frame=single]
-A INPUT -p udp --dport 5000:5001 -j ACCEPT
\end{lstlisting} 
\vspace*{-2mm}

Notwendige Tastatureingaben: \\

Verwenden Sie zum Speichern der Datei die Tastenkombination \textit{STRG + o} und bestätigen Sie danach mit \textit{ENTER}. Mit der Tastenkombination \textit{STRG + x} schließen Sie die Datei. 



%-------------------------------------------------------------------%

\section{vServer}


Folgen Sie den Anweisungen im Benutzerhandbuch, um auf den virtuellen Server zuzugreifen.

Im derzeitigen Zustand ist das Programm sowie dessen Interaktion mit der Datenbank und der Webseite auf dem virtuellen Server bereits funktionstüchtig. Eine Änderung der Konfiguration ist unter Umständen nur notwendig bei Modifikationen an der MicroAutoBox II oder des Embedded-PCs. Eine jederzeit aktuelle Installations- bzw. Konfigurationsanleitung finden Sie unter:

\vspace*{4mm}
\begin{lstlisting}[frame=single]
https://github.com/fooWander/tui_swp_formula.git
\end{lstlisting} 
\vspace*{-2mm}

\textbf{Punkt 1)} \\

Standardmäßig wird das Programm automatisch beim Start des PCs gestartet.Dies erfolgt über einen Eintrag in der Datei \textit{rc.local}.
Die Datei \textit{rc.local} befindet sich unter folgendem Verzeichnis:

\vspace*{4mm}
\begin{lstlisting}[frame=single]
/etc/rc.local
\end{lstlisting} 
\vspace*{-2mm}

\textbf{Punkt 2)} \\

Das erwähnte Programm heißt \textit{vserver} und liegt ausgehend vom Rootnutzer im nachfolgenden Verzeichnis:

\vspace*{4mm}
\begin{lstlisting}[frame=single]
~/tui_swp_formula/bin
\end{lstlisting} 
\vspace*{-2mm}

\subsection{Netzwerkanpassungen des Programms}

%\textbf{Punkt 3)} \\

Änderungen von IP-Adressen oder Portnummern können Sie wie folgt durchführen:

\vspace*{4mm}
\begin{lstlisting}[frame=single]
cd ~/tui_swp_formula/src

nano common.h
\end{lstlisting} 
\vspace*{-2mm}

Die nun dargestellte Datei editieren Sie entsprechend Ihren Änderungen. Genauere Anweisungen hierzu finden Sie als Kommentare vor.\\

Nachfolgend notwendige Tastatureingaben: \\

Verwenden Sie zum Speichern der Datei die Tastenkombination \textit{STRG + o} und bestätigen Sie danach mit \textit{ENTER}. Mit der Tastenkombination \textit{STRG + x} schließen Sie die Datei. 

Um das Programm nach vorgenommenen Netzwerk-Änderungen neu zu starten ist folgende Vorgehensweise einzuhalten:

\vspace*{4mm}
\begin{lstlisting}[frame=single]
killall vserver

cd ~/tui_swp_formula/src
	
make vserver %*–B*)

~/tui_swp_formula/bin/vserver 1>/dev/null &
\end{lstlisting} 
\vspace*{-2mm}

\subsection{Installation und Einrichtung der benötigten Komponenten bei Verwendung eines neuen Betriebssystems}

\vspace*{4mm}
\begin{lstlisting}[frame=single]
yum update

yum install ntp git ssh nano

yum groupinstall "Development Tools"
\end{lstlisting} 
\vspace*{-2mm}

\textbf{Punkt 1)} \\

Eine Zeitsynchronisation ist wie folgt durchzuführen: \\
Den NTP-Dienst aktivieren Sie mit: 

\vspace*{4mm}
\begin{lstlisting}[frame=single]
chkconfig ntpd on
\end{lstlisting} 
\vspace*{-2mm}

\newpage

Die Zeit können Sie nun mit dem Zeitserver \textit{0.europe.pool.ntp.org} wie folgt synchronisieren:

\vspace*{4mm}
\begin{lstlisting}[frame=single]
ntpdate 0.europe.pool.ntp.org
\end{lstlisting} 
\vspace*{-2mm}

Zuletzt wird der NTP-Dienst über folgende Eingabe gestartet: 

\vspace*{4mm}
\begin{lstlisting}[frame=single]
/etc/init.d/ntpd start
\end{lstlisting} 
\vspace*{-2mm}

Die Zeit kann über nachfolgenden Befehl kontrolliert werden:

\vspace*{4mm}
\begin{lstlisting}[frame=single]
date
\end{lstlisting} 
\vspace*{-2mm}

\textbf{Punkt 2)} \\

Die aktuelle Programmversion können Sie wie folgt über eine bestehende Internetverbindung herunterladen und installieren:

\vspace*{4mm}
\begin{lstlisting}[frame=single]
cd ~/
 
git clone git://github.com/fooWander/tui_swp_formula.git

cd tu-ilmenau/src/

make vserver %*-B*)
\end{lstlisting} 
\vspace*{-2mm}

\textbf{Punkt 3)} \\

Standardmäßig wird das Programm automatisch beim Start des PCs gestartet. Die Datei \textit{rc.local} befindet sich unter folgendem Verzeichnis:

\vspace*{4mm}
\begin{lstlisting}[frame=single]
/etc/rc.local
\end{lstlisting} 
\vspace*{-2mm}

\textbf{Punkt 4)} \\

Das Programm wird automatisch bei Systemstart ausgeführt, wenn Sie folgende Vorgehensweise beachten:

\vspace*{4mm}
\begin{lstlisting}[frame=single]
echo ~/tui_swp_formula/bin/vserver >> /etc/rc.local
\end{lstlisting} 
\vspace*{-2mm}

\newpage









%%%%% TEXT DANIEL %%%%%

Die nachfolgenden Komponenten sorgen einerseits für den Empfang und die Dekodierung der Datenpakete vom Embedded-PC, als auch für das Einfügen dieser Pakete in die Datenbank sowie das Löschen veralteter Datensätze aus dieser. \\
Nachfolgend wird die korrekte Einrichtung dieser Komponenten erläutert.

\subsection{Installation der Boost-Bibliotheken}

Boost stellt C++-Bibliotheken zur Verfügung, welche auf dem vServer zwingend benötigt werden.
Dazu muss auf beiden Systemen der folgende Befehl erfolgreich ausgeführt werden:

\vspace*{4mm}
\begin{lstlisting}[frame=single]
yum install boost boost-devel %*-y*)
\end{lstlisting} 
\vspace*{-2mm}
    
    
Zur Installation über die Kommandozeile werden \textit{Administratorrechte} benötigt.

\subsection{Installation des MySQL-Connector/C++}

Der MySQL Connector/C++ ist zwingend erforderlich, damit das C++-Programm des virtuellen Servers auf die Datenbank zugreifen kann. Für die Installation benötigen Sie \textit{root}-Rechte, die folgenden Befehle werden in die Kommandozeile eingegeben.
Zuerst einmal muss sichergestellt werden, dass alle im weiteren Verlauf verwendeten Programme vorhanden sind:

\vspace*{4mm}
\begin{lstlisting}[frame=single]
yum install bzr boost\_devel cmake mysql-devel %*–y*)
\end{lstlisting} 
\vspace*{-2mm}

Anschließend kann der aktuelle Quellcode bezogen werden:

\vspace*{4mm}
\begin{lstlisting}[frame=single]
%*bzr branch lp:~mysql/mysql-connector-cpp/trunk ./mysql-connector-cpp*) 
\end{lstlisting} 
\vspace*{-2mm}


In das soeben Verzeichnis wechseln und die Makefile erstellen:

\vspace*{4mm}
\begin{lstlisting}[frame=single]
cd mysql-connector-cpp

cmake . 
\end{lstlisting} 
\vspace*{-2mm}
  

Erstellung der Bibliotheken und Installation der Header-Dateien: 

\vspace*{4mm}
\begin{lstlisting}[frame=single]
make clean 

make 

make install  
\end{lstlisting} 
\vspace*{-2mm}
  

Die Kompilierung des Programms kann einige Minuten in Anspruch nehmen. Für weitere Informationen, Testanleitungen und ggf. Fehlerbehandlung verweisen wir auf die MySQL-Dokumentation: \\

\vspace*{4mm}
\begin{lstlisting}[frame=single]
 http://dev.mysql.com/doc/refman/5.1/en/connector-cpp-installation-source-unix.html
 \end{lstlisting} 
\vspace*{-2mm}
 
 \subsection{Einstellungen über die Konfigurationsdatei}
 %Achtung, noch nicht final?
Die Konfigurationsdatei \textit{Konfiguration.conf} befindet sich im Installationsverzeichnis. Sie wird von dem Programm während der Laufzeit ausgewertet, damit Programmparameter geändert werden können, ohne das Programm neu kompilieren zu müssen.\\
In dieser Konfigurationsdatei werden die Zugangsdaten für die Datenbank, die Anzahl der in der Datenbank zu haltenden Datensätze sowie interne Parameter des Programms (IP-Adressen, Portnummern) festgesetzt.
Sollten Änderungen an diesen Parametern (beispielsweise bei der Anzahl) nötig sein, ist darauf zu achten, dass der Syntax der Datei erhalten bleibt und die Änderungen sowohl auf dem vServer als auch auf dem Embedded-PC eingepflegt werden.
Der jeweilige Parameter wird aus der auf die Beschreibung bzw. Nennung des Parameters folgenden Zeile ausgelesen. 
Diese darf nicht leer sein, auf Leerzeichen und zusätzliche Zeilenumbrüche sollte verzichtet werden.
Alle Zeilen, die mit einem Doppelkreuz (\#) beginnen, dürfen nicht verändert werden, da sonst das Programm die Parameter nicht mehr erkennt. \\

Beispielhafter Ausschnitt aus der Konfigurationsdatei:\\

\vspace*{4mm}
\begin{lstlisting}[frame=single]
\#Hostadresse der Datenbank
localhost

\#Name des Datenbankbenutzers
telemetrie

\#Passwort des Datenbankbenutzers
fakePW

\#Datenbankname
telemetrie 
\end{lstlisting} 
\vspace*{-2mm}

    


\textit{Wichige Anmerkungen:} \\ \\
Stellen Sie sicher das alle Parameter korrekt gesetzt sind! Ansonsten kann ein störungsfreier Betrieb des Programms nicht gewährleistet werden!\\
Des Weiteren ist darauf zu achten, das nur befugte Personen diese Datei einsehen können, da sie das Zugangspasswort zur Datenbank enthält!\\
Bei Fehler kann gegebenenfalls die Logdatei (log.txt im Installationsverzeichnis) nach Informationen durchsucht werden.

\subsection{Installation des Cronjobs}

Um die Datenbank in regelmäßigen Abständen auf ihre Größe zu prüfen und gegebenenfalls das Löschen von Einträgen zu veranlassen, wird ein Cronjob angelegt, durch den das eigenständige Programm \textit{StartResizeDB} in periodischen Zeitintervallen vom System aufgerufen wird.
Die Einrichtung des Cronjobs erfolgt manuell über die Kommandozeile des virtuellen Servers. Hierzu werden \textit{root}-Rechte benötigt.\\

Öffnen der Datei \textit{/ect/crontab} mit einem Editor (zum Bsp: \textit{nano}, \textit{vi}):\\

\vspace*{4mm}
\begin{lstlisting}[frame=single]
 vi /etc/crontab  
\end{lstlisting} 
\vspace*{-2mm}
   

Hier muss zuerst die Umgebungsvariable \textit{,,HOME"} gesetzt werden, damit das Programm nicht unter dem Wurzelverzeichnis ausgeführt wird. Der anzugebende Pfad ist abhängig vom Installationsverzeichnis und sollte auf den Überordner des Programms \textit{StartResizeDB} verweisen.
 
\vspace*{4mm}
\begin{lstlisting}[frame=single]
HOME=/Pfad/zum/Installationsverzeichnis/ 
\end{lstlisting} 
\vspace*{-2mm}


Nun wird an das Ende der Datei folgende Zeile hinzugefügt, wobei der Pfad zu ersetzen ist:

\vspace*{4mm}
\begin{lstlisting}[frame=single]
0 * * * * root /Pfad/zur/StartResizeDB 2>&1 
\end{lstlisting} 
\vspace*{-2mm}


Der anzugebende Pfad ist abhängig vom Installationsverzeichnis und sollte auf die Datei \textit{StartResizeDB} verweisen.
Damit wird das angegebene Programm zu jeder vollen Stunde ausgeführt, \textit{,,root"} ist hier der ausführende Benutzer. Der angegebene Benutzer sollte Rechte zum Lesen und Schreiben auf alle Projektdateien besitzen.\\
\textit{"`2 \textgreater \& 1"} bedeutet, dass die Cron-Standardausgabe benutzt wird.
Es ist auf genaue Einhaltung des Syntax zu achten.\\
Beim anschließenden Speichern und Schreiben der Datei wird der Cronjob automatisch installiert.








%-------------------------------------------------------------------%

\section{Datenbanken}

Auf die Installation der Datenbanken wird im nachfolgenden Kapitel eingegangen. Hierzu werden bestimmte Skripte ausgeführt, die genaue Verwendung davon dort, wenn sie den Einrichtungsanweisungen der Webseite folgen.

%-------------------------------------------------------------------%



\newpage

% Text von Thomas eingepflegt, gez. Sebastian

\section{Webseite}

Zur uneingeschränkten Nutzung der Software müssen einige Voraussetzungen erfüllt sein: 

\begin{itemize}

\item Webserver 

\begin{itemize}

\item PHP Version 5.3 oder höher
\item mindestens eine (idealerweise zwei) MySQL-Datenbank (MySQL Version 5.3 oder höher)
\item SMTP-Server mit Authentifizierung
\item X MB freien Speicherplatz für Webseite

\end{itemize}

\end{itemize}

Im Folgenden werden alle nötigen Installationsschritte für die gesamte Software sowie die entsprechenden Konfigurationseinstellungen erläutert, welche zu Beginn getroffen werden müssen. \\

\textit{Webseite - Konfiguration} \\

Alle ausgelieferten Verzeichnisse und Dateien müssen in das Stammverzeichnis (s. Beschreibung Ihres Hostingangebotes) hochgeladen werden. Bevor Sie dies jedoch tun, müssen sie die Datei \textit{includes/config.php} anpassen. \\ 

\begin{figure}[h]
\centering
\includegraphics[scale = 0.50]{website_config}
\caption[Auschnitt aus der config.php - Datei]{Auschnitt aus der config.php - Datei}
\label{websiteconfig}
\end{figure} 

\newpage

Für Sie sind lediglich sechs Zeilen wichtig: 

\begin{itemize}[leftmargin=*]

\item \textit{\$dbhost}: Diese Einstellung ist auf "`localhost"\ gestellt. In den meisten Fällen ist dies die Standardeinstellung. Sollten Sie von Ihrem Provider explizit andere Angaben erhalten haben, dann ändern Sie dieses Feld. Sollten Sie keine Informationen erhalten haben, wird mit hoher Wahrscheinlichkeit "`localhost"\ die richtige Wahl sein. Sollte es zu Problemen kommen, setzen Sie sich bitte mit Ihrem Provider in Verbindung.


\item \textit{\$dbuname}: Hier fügen Sie in einfachen Anführungszeichen den Ihnen vom Provider mitgeteilte Zugangsnamen für den Datenbankserver ein.


\item \textit{\$dbpass}: Hier fügen Sie entsprechend das an Sie vergebene Passwort für die Datenbank ein.

\item \textit{\$dbname\_fd}: Sollten Sie vom Provider bereits eine Datenbank erhalten haben, so fügen Sie hier den Namen der Datenbank ein. Wenn Sie vollen Zugriff auf den Datenbankserver haben und selber Datenbanken anlegen können, so steht es Ihnen frei, wie Sie die Datenbank der Fahrzeugdaten bezeichnen möchten.


\item \textit{\$dbname\_ud}: Sollten Sie vom Provider bereits eine Datenbank erhalten haben, so fügen Sie hier den Namen der Datenbank ein. Wenn Sie vollen Zugriff auf den Datenbankserver haben und selber Datenbanken anlegen können, so steht Ihnen die Wahl frei, wie Sie die Datenbank der Nutzerdaten bezeichnen möchten.

\textbf{Anmerkung:} Beide Datenbanknamen können identisch sein, z.B. wenn Sie von Ihrem Provider nur eine Datenbank erhalten haben sollten. 

\item \textit{\$mail}: Hier tragen Sie bitte die E-Mail-Adresse des Vorstandes ein. Alle eingehenden Registrierungsanfragen werden an diese E-Mail-Adresse weitergeleitet. Diese kann auch nach der Installtion noch angepasst werden. Alle anderen Werte müssen ab dem Beenden der Installation unverändert bleiben um die volle Funktionsfähigkeit gewährleisten zu können.

\end{itemize}

\newpage


\textit{Webseite - Installation} \\

Nachdem die Konfiguration erfolgreich durchgeführt wurde, öffnen Sie die Webseite in einem Browser und geben manuell den Pfad \textit{,,install.php"} zusätzlich zur Adresse der Webseite in die Suchleiste ein.\\
Pfadbeispiel: \textit{team-starcraft.de/swp\_formula\_student/install.php} - dabei ist \textit{swp\_formula\_student} das Hauptverzeichnis der Webseite. \\
 Danach füllen Sie das entsprechende Formular aus und schicken dieses ab. 
 Anschließend werden alle nötigen Datenbanken und Tabellen erzeugt, sowie die Nutzergruppen und der Vorstandsaccount eingerichtet.

\begin{figure}[h]
\centering
\includegraphics[scale = 0.50]{install}
\caption[Einrichtung des Vorstand - Accounts]{Einrichtung des Vorstand - Accounts}
\label{installvorstand}
\end{figure} 


Sollten Sie von Ihrem Provider bereits Datenbanken erhalten haben, tritt nach der Installation ein Fehler mit der Meldung auf, dass die zu erstellende Datenbank bereits existiert. Dies ist normal und kein Grund zur Sorge. \\ \\

Die Installation ist nun vollständig. Bitte löschen Sie die \textit{install.php} unverzüglich vom Server um Missbrauch zu vermeiden. Sie können Sich nun mit den angegeben Zugangsdaten einloggen.




%----------------- Bedienung des Service-Interfaces ----------------%

\chapter{Bedienung des Service-Interfaces}

\section{Embedded-PC}

\subsection{Zugriff auf den Embedded-PC}

\textbf{Punkt 1)} \\

\textit{Lokaler Zugriff} \\

Bei Anschluss der Stromversorgung startet der PC automatisch.
Nach erfolgreichem Start werden Sie gebeten folgende Daten einzugeben:

\vspace*{4mm}
\begin{lstlisting}[frame=single]
Benutzername:		root

Passwort:		    swp2013

\end{lstlisting} 
\vspace*{-2mm}

Das Root-Passwort kann durch den nachstehenden Befehl geändert werden:

\vspace*{4mm}
\begin{lstlisting}[frame=single]
passwd
\end{lstlisting} 
\vspace*{-2mm}

%\textbf{Punkt x)} \\

Remotezugriff (SSH/SFTP-Verbindung) \\

Der Embedded-PC ist über eine SSH-Verbindung zu erreichen. 
Benutzername und Passwort sind unter Punkt x zu entnehmen. 
Die Verbindungsinformationen dazu sind:

\vspace*{4mm}
\begin{lstlisting}[frame=single]
%*Port:  		22 *)
%*IP-Adresse: 	192.168.0.2 (bzw. siehe Punkt y bei Zugriff über eine öffentliche IP)*)

\end{lstlisting} 
\vspace*{-2mm}

Der konkrete Befehl in einem Terminal lautet mit den Beispieldaten:

\vspace*{4mm}
\begin{lstlisting}[frame=single]
%*ssh root@192.168.0.2 –p22*)
\end{lstlisting} 
\vspace*{-2mm}

\textbf{Punkt 2)} \\

Die öffentliche IP-Adresse ermitteln Sie z.B. durch den Befehl:	

\vspace*{4mm}
\begin{lstlisting}[frame=single]
%*curl http://ip.skittel.de*)
\end{lstlisting} 
\vspace*{-2mm}

\newpage

\textbf{Punkt 3)} \\

\textit{Installation von Programmupdates} \\

Updates werden vorerst manuell mittels des Versionsverwaltungstools Git bezogen. Dazu gehen Sie folgendermaßen vor:

\vspace*{4mm}
\begin{lstlisting}[frame=single]
killall embpc

cd ~/tui_swp_formula

git pull && cd src

make embpc %*–B*)

\end{lstlisting} 
\vspace*{-2mm}

\textbf{Punkt 4)} \\

\textit{Starten bzw. Neustarten des Programms} 

\vspace*{4mm}
\begin{lstlisting}[frame=single]
killall embpc

~/tui_swp_formula/bin/embpc 1>/dev/null &
\end{lstlisting} 
\vspace*{-2mm}





\section{vServer}

\subsection{Zugriff auf den virtuellen Server}		

\textit{Remotezugriff (SSH/SFTP-Verbindung)} \\

Der virtuelle Server ist über eine SSH-Verbindung zu erreichen. \\
(Servername: \textit{s16929463.onlinehome-server.info}) \\
Die Verbindungsinformationen dazu sind:

\vspace*{4mm}
\begin{lstlisting}[frame=single]
%*Port:  		22*)

%*IP-Adresse: 	87.106.17.165*) 

%*Benutzername:		root*)

%*Passwort:			aYYKn9Td*)
\end{lstlisting} 
\vspace*{-2mm}

Das Root-Passwort kann durch den nachstehenden Befehl geändert werden:

\vspace*{4mm}
\begin{lstlisting}[frame=single]
passwd
\end{lstlisting} 
\vspace*{-2mm}

\newpage

Der konkrete Befehl in einem Terminal lautet mit den Beispieldaten:	

\vspace*{4mm}
\begin{lstlisting}[frame=single]
%*ssh root@87.106.17.165  –p22*)
\end{lstlisting} 
\vspace*{-2mm}

\subsection{Installation von Programmupdates}

Updates werden vorerst manuell mittels des Versionsverwaltungstools Git bezogen. Dazu gehen Sie folgendermaßen vor:

\vspace*{4mm}
\begin{lstlisting}[frame=single]
killall vserver

cd ~/tui_swp_formula

git pull && cd src

make vserver %*–B*)
\end{lstlisting} 
\vspace*{-2mm}
%
%\textbf{Punkt n)} \\

\textit{Starten bzw. Neustarten des Programms}

\vspace*{4mm}
\begin{lstlisting}[frame=single]
killall vserver

~/tui_swp_formula/bin/vserver 1>/dev/null &
\end{lstlisting} 
\vspace*{-2mm}

%\newpage











\section{Webseite}

\subsection{Startseite / Verwaltung}

Über die Startseite loggen Sie sich mit den Zugangsdaten mit denen Sie sich registriert haben ein. 
Ob Ihr Account bereits freigeschaltet wurde, erfahren Sie vom Vorstand. 
Nach dem Einloggen werden hier allgemeine Informationen dargestellt, wie z.B. eine Liste der sich zur Zeit online befindlichen Nutzer und eine Exportfunktion zum Extrahieren der Fahrzeugdaten aus der Datenbank. 
Als Vorstand erhalten Sie zudem noch eine Liste sämtlicher registrierter Nutzer und eine Möglichkeit Nutzer freizuschalten bzw. zu löschen und Rechtegruppen zu vergeben oder zu ändern. \\

\textbf{Anmerkung:} Bitte verwenden Sie für die Registrierung im Service Interface entweder Ihre E-Mail-Adresse vom Team Starcraft, eine TU Ilmenau E-Mail-Adresse oder eine 1und1-Adresse um möglichen Problemen aus dem Weg zu gehen.

\subsection{Nutzerverwaltung (Vorstand)}

Wenn Sie einen Nutzer bearbeiten wollen, muss stets eine der beiden Radioboxen aktiviert sein und im Textfeld seine ID-Nummer eingetragen werden. Möchten Sie einen Nutzer löschen, so wählen sie "`Löschen"\ , möchten Sie ihn jedoch aktivieren oder bearbeiten, so wählen Sie "`Aktivieren"\ aus. Mittels des Dropdown-Menüs können Sie dem Benutzer eine Rechtegruppe zuweisen. \\ \\
\textbf{Achtung:} Sie können sich nicht selbst löschen, dies muss ein anderer Nutzer mit Vorstandsrechten für Sie erledigen!

\subsection{CSV-Export}

Durch Auswahl einzelner Checkboxen können Sie sich die Daten des Fahrzeugs als \\ CSV-Datei herunterladen. Aus technischen Gründen kann es beim Auswählen mehrerer Boxen dazu kommen, dass Datensätze fehlen. Dies können Sie vermeiden, indem sie die Tabellen einzeln exportieren.

\subsection{Passwort ändern}

Diese Seite ermöglicht es Ihnen Ihr Passwort - beispielsweise nach einem Reset - zu ändern. Sie erreichen diese über die unter dem Hauptmenü befindlichen Link "`Passwort ändern".

\subsection{Passwort vergessen}

Auf der Startseite befindet sich ein Link "`Passwort vergessen". Klicken Sie auf ihn und geben Sie ihre Emailadresse ein. Bei erfolgreicher Änderung des Passworts erhalten Sie das neue Passwort per Mail zugeschickt. Andernfalls erscheint eine Fehlermeldung.

\subsection{Menüleiste}

Über die Menüleiste können Sie die einzelnen Unterseiten aufrufen. Jede Unterseite stellt eine andere Gruppe von Fahrzeuginformationen dar (s. Pflichtenheft). \\ \\

Es ist zu Empfehlen sich nach jedem Besuch der Seite wieder auszuloggen.


%----------------------  Abbildungsverzeichnis  --------------------%

\listoffigures

\addcontentsline{toc}{chapter}{Abbildungsverzeichnis}


\end{document}