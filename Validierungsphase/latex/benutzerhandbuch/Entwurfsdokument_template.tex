\documentclass[fontsize = 12pt, paper = a4]{scrreprt} 

\setlength{\parindent}{0pt}
\usepackage[english,ngerman]{babel}
\usepackage[utf8]{inputenc} 
\usepackage{enumerate}
\usepackage{amssymb,amsmath}

%------------ Überschriften verkleinern und hochsetzen ----------%

%\makeatlettern
%\renewcommand*\@makechapterhead[1]{%
%{\parindent \z@ \raggedright \normalfont
%\LARGE\bfseries
%\ifnum \c@secnumdepth >\m@ne
%\thechapter\space
%\fi
%#1\par\nobreak
%\vskip 20\p@
%}} 

% ------------------------ Blattlayout- -------------------------%

\usepackage {geometry}   
\geometry   {left     = 2.5cm,
             right    = 2.5cm, 
             top      = 1.5cm,
             bottom   = 1.5cm,
             includehead, includefoot}
             
% ------------------------ Seitenstil ---------------------------%           

% Umdefinieren von Befehlen zur Vermeidung von Bugs:

\renewcommand*{\chapterpagestyle}{scrheadings} 
\renewcommand*{\chapterheadstartvskip}{\vspace*{-\topskip}}

% Gestaltung der Kopf- und Fußzeile:

\pagenumbering{arabic}
            
\usepackage[automark]{scrpage2}
\automark[chapter]{section}
\pagestyle{scrheadings} 
\ohead[\pagemark]{\pagemark}
\setlength{\footskip}{5mm} 

\clearscrheadfoot
\lohead{Benutzerhandbuch}
\rohead{\headmark}
\lofoot{Softwareprojekt TU Ilmenau SS 2013}
\rofoot{\pagemark}

% Kopf- und Fußzeilenlinie:

\setheadsepline{.6pt} % Linie für Kopfzeile
\setfootsepline{.6pt} % Linie für Fußzeile 

% Für Unterstreichungen:

\usepackage[normalem]{ulem}

% Buchstabenglättung am Rand:
  
\usepackage {microtype} 

% Bildunterschriften zentrieren:

%\usepackage{caption}
%\captionsetup{margin=10pt,font=small,labelfont=bf, justification = centering}

%-------------------------------------------------------------------%

% Für die Einbindung von Bildern:

\usepackage[pdftex]{graphicx} % .pdf, .png oder .jpg möglich!
\usepackage{rotating}         % Grafiken rotieren

% Nutzung in drei Umgebungen möglich:

% (1) \begin{turn}{Winkel} ...  \end{turn}
% (2) \begin{sideways} ... \end{sideways} 90° im math. pos. Sinn
% (3) \begin{rotate}{Winkel} ... \end{rotate} 
%     ---> 90° im math. pos. Sinn, allerdings keine Platzreservierung 

\usepackage{wrapfig}
%\usepackage{picins}   % Textumflossene Grafiken
\usepackage{subfigure}
\usepackage{floatflt}
\usepackage[justification=centering]{caption}

%-------------------------------------------------------------------%
 
% Packete für Tabellen:

\usepackage{booktabs}
\usepackage{array}    % optional
\usepackage{tabularx} % optional

\usepackage[font=footnotesize,labelfont=bf,singlelinecheck=false,
            format=plain,,justification=justified,indention=0cm]                     {caption} 

\usepackage{setspace}

%----------------  Anfang des Dokuments ------------------%

\begin{document}

%*******************************************************************%

% Entwurf Titelseite:

\titlehead{\begin{center}
\textbf{\Huge Entwurfsdokument}
\end{center}}
		   
\title{Service-Interface \\ für ein Formula-Student-Fahrzeug}

\subtitle{Technische Universität Ilmenau \\
		  Softwareprojekt SS 2013 \\ Gruppe 19}			
		
\author{Christian Boxdörfer \\ Thomas Golda \\ Daniel Häger \\ 
		David Kudlek \\  Tom Porzig \\ Tino Tausch \\ 
		Tobias Zehner \\ Sebastian Zehnter}
		
\date{Hier Datum einfügen}	 
	  
\publishers{betreut durch \\ \vspace{1cm} Dr. Heinz-Dietrich Wuttke, TU Ilmenau \\ Oliver Dittrich, fachlicher Betreuer Team StarCraft e.V.}

\maketitle		

%*******************************************************************%

% --------------------- Inhaltsverzeichnis -----------------------%

\begin{spacing}{0.86} 
\tableofcontents
%\setcounter{secnumdepth}{4} % Tiefere Gliederungsebene  
\setcounter{tocdepth}{4} % Anzeige bis Gliederungsstufe 4
%\addtocontents{toc}{\protect\enlargethispage{2\baselineskip}} 
\end{spacing}


\newpage % Seitenumbruch

%--------------------------  Einleitung  ---------------------------%

\chapter{Einleitung}

%----- Installation und Konfiguration des Service Interfaces -------%

\chapter{Installation und Konfiguration des Service Interfaces}

\section{MicroAutoBox II}

\section{Embedded-PC}

\section{vServer}

\section{Datenbanken}

\subsection{Fahrzeugdatenbank}

\subsection{Benutzerdatenbank}

\section{Webseite}

%----------------- Bedienung des Service-Interfaces ----------------%

\chapter{Bedienung des Service-Interfaces}


\end{document}